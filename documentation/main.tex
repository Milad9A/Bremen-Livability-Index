% !TEX root = main.tex
% ============================================================
% Bremen Livability Index – Projektdokumentation
% Hochschule Bremen · Geodatenverarbeitung · WS 2025/26
% Autor: Milad Awad (5358834)
% ============================================================
\documentclass[
  11pt,
  a4paper,
  parskip=half-,
  bibliography=totoc
]{scrreprt}

% ── Sprache & Kodierung ──────────────────────────────────────
\usepackage[ngerman]{babel}
\usepackage[utf8]{inputenc}
\usepackage[T1]{fontenc}
\usepackage{csquotes}
\usepackage{lmodern}

% ── Seitenränder ─────────────────────────────────────────────
\usepackage[
  left=2cm,
  right=2cm,
  top=2cm,
  bottom=2cm
]{geometry}

% ── Zeilenabstand ────────────────────────────────────────────
\usepackage{setspace}
\singlespacing

% ── Typografie ───────────────────────────────────────────────
\usepackage{microtype}
\usepackage{xurl}

% ── Mathematik ───────────────────────────────────────────────
\usepackage{amsmath}
\usepackage{amssymb}

% ── Tabellen ─────────────────────────────────────────────────
\usepackage{booktabs}
\usepackage{longtable}
\usepackage{tabularx}
\usepackage{array}

% ── Grafiken ─────────────────────────────────────────────────
\usepackage{graphicx}
\graphicspath{{images/}}

% ── Farben ───────────────────────────────────────────────────
\usepackage[dvipsnames,table]{xcolor}

% ── Quellcode-Listings ───────────────────────────────────────
\usepackage{listings}
\lstset{
  basicstyle=\ttfamily\small,
  keywordstyle=\color{blue}\bfseries,
  commentstyle=\color{gray},
  stringstyle=\color{OliveGreen},
  numbers=left,
  numberstyle=\tiny\color{gray},
  numbersep=8pt,
  frame=single,
  breaklines=true,
  breakatwhitespace=false,
  showspaces=false,
  showstringspaces=false,
  tabsize=4,
  captionpos=b,
  xleftmargin=1.5em,
  framexleftmargin=1em,
  aboveskip=0.6em,
  belowskip=0.6em
}
\lstdefinestyle{python}{
  language=Python,
  morekeywords={self, True, False, None, async, await}
}
\lstdefinestyle{sql}{
  language=SQL,
  morekeywords={GEOGRAPHY, GIST, SCHEMA, CASCADE, TRUNCATE}
}
\renewcommand{\lstlistlistingname}{Listingverzeichnis}
\renewcommand{\lstlistingname}{Listing}

\lstdefinestyle{json}{
  string=[s]{"}{"},
  comment=[l]{:\ "},
  morecomment=[l]{:"},
  literate=
    *{0}{{{\color{purple}0}}}{1}
    {1}{{{\color{purple}1}}}{1}
    {2}{{{\color{purple}2}}}{1}
    {3}{{{\color{purple}3}}}{1}
    {4}{{{\color{purple}4}}}{1}
    {5}{{{\color{purple}5}}}{1}
    {6}{{{\color{purple}6}}}{1}
    {7}{{{\color{purple}7}}}{1}
    {8}{{{\color{purple}8}}}{1}
    {9}{{{\color{purple}9}}}{1}
}

% ── Hyperlinks ───────────────────────────────────────────────
\usepackage[
  colorlinks=true,
  linkcolor=NavyBlue,
  citecolor=OliveGreen,
  urlcolor=BrickRed,
  pdfauthor={Milad Awad},
  pdftitle={Bremen Livability Index},
  pdfsubject={Geodatenverarbeitung},
  pdfkeywords={GIS, PostGIS, Livability, Bremen, OpenStreetMap}
]{hyperref}

% ── Literatur ────────────────────────────────────────────────
\usepackage[
  backend=bibtex,
  style=authoryear,
  sorting=nyt,
  maxcitenames=2,
  maxbibnames=99,
  giveninits=true,
  uniquename=init,
  dashed=false
]{biblatex}
\addbibresource{references.bib}

% ── Glossar / Abkürzungen ────────────────────────
\usepackage[acronym,toc]{glossaries}
\makeglossaries
\renewcommand*{\glsgroupskip}{}

\newacronym{gis}{GIS}{Geoinformationssystem}
\newacronym{osm}{OSM}{OpenStreetMap}
\newacronym{api}{API}{Application Programming Interface}
\newacronym{crs}{CRS}{Coordinate Reference System (Koordinatenreferenzsystem)}
\newacronym{wgs84}{WGS\,84}{World Geodetic System 1984}
\newacronym{utm}{UTM}{Universal Transverse Mercator}
\newacronym{orm}{ORM}{Object-Relational Mapping}
\newacronym{ci}{CI/CD}{Continuous Integration \slash Continuous Deployment}
\newacronym{bloc}{BLoC}{Business Logic Component}
\newacronym{cors}{CORS}{Cross-Origin Resource Sharing}
\newacronym{rest}{REST}{Representational State Transfer}
\newacronym{sql}{SQL}{Structured Query Language}

% ── Kopf- und Fußzeilen ─────────────────────────────────────
\usepackage[headsepline]{scrlayer-scrpage}
\pagestyle{scrheadings}
\automark[section]{chapter}

% ── Sonstige ─────────────────────────────────────────────────
\usepackage{enumitem}
\setlist{nosep,leftmargin=*}
\setlist[description]{style=nextline,nosep,leftmargin=1em}
\usepackage{float}
% Kompaktere Kapitelüberschriften
\RedeclareSectionCommand[
  beforeskip=1em,
  afterskip=0.5em
]{chapter}
\RedeclareSectionCommand[
  beforeskip=0.8em,
  afterskip=0.3em
]{section}
\RedeclareSectionCommand[
  beforeskip=0.5em,
  afterskip=0.2em
]{subsection}
% ==============================================================
%  TITELSEITE
% ==============================================================
\begin{document}

\begin{titlepage}
  \centering
  \vspace*{1cm}

  \includegraphics[width=0.45\textwidth]{hsb_logo}

  \vspace{1.5cm}

  {\LARGE\textbf{Bremen Livability Index}}\\[0.8em]
  {\large\textit{Ein geodatenbasiertes Bewertungssystem\\
    für die Lebensqualität in Bremen}}

  \vspace{2.5cm}

  {\large Projektdokumentation im Modul}\\[0.4em]
  {\Large\textbf{Geodatenverarbeitung}}\\[0.4em]
  {\large Wintersemester 2025\,/\,2026}

  \vspace{2cm}

  \begin{tabular}{rl}
    \textbf{Autor:}        & Milad Awad              \\
    \textbf{Matrikel-Nr.:} & 5358834                 \\[0.8em]
    \textbf{Dozent:}       & Dr.-Ing. Christian Seip \\
  \end{tabular}

  \vfill

  {\large Bremen, \today}
\end{titlepage}

% ==============================================================
%  VERZEICHNISSE
% ==============================================================
\pagenumbering{roman}

\tableofcontents
\clearpage

\begingroup
\let\clearpage\relax
\let\cleardoublepage\relax
\addcontentsline{toc}{chapter}{\listfigurename}
\listoffigures
\addcontentsline{toc}{chapter}{\listtablename}
\listoftables
\addcontentsline{toc}{chapter}{\lstlistlistingname}
\lstlistoflistings
\endgroup
\clearpage

\printglossary[type=\acronymtype, title={Abkürzungsverzeichnis}]
\clearpage

% ==============================================================
%  KAPITEL
% ==============================================================
\pagenumbering{arabic}

% ============================================================
\chapter{Einleitung}
\label{ch:einleitung}
% ============================================================

\section{Motivation}

Die Wahl des Wohnortes ist eine der weitreichendsten Entscheidungen im Alltag.
Faktoren wie die Nähe zu Grünflächen, die Erreichbarkeit von Nahversorgung und
öffentlichem Nahverkehr, aber auch potenzielle Belastungen durch Lärm, Verkehr
oder Industrieanlagen beeinflussen die Lebensqualität eines Standortes erheblich.
Obwohl zahlreiche Geodaten zu diesen Aspekten frei verfügbar sind -- insbesondere
über \acrfull{osm} und den Unfallatlas des Statistischen Bundesamtes -- fehlt es
an Werkzeugen, die diese heterogenen Datenquellen standortbezogen aggregieren und
in einem leicht verständlichen Index zusammenfassen.

Internationale Lebensqualitätsrankings wie der \textit{Global Liveability Index}
der Economist Intelligence Unit \parencite{economist_gla} oder der \textit{Quality
      of Living Index} von Mercer \parencite{mercer2019} bewerten Städte auf nationaler
oder globaler Ebene, bieten jedoch keine Auflösung auf Stadtteil- oder gar
Adressebene. Hier setzt das vorliegende Projekt an.

\section{Zielsetzung}

Ziel des Projekts \textbf{Bremen Livability Index} (\textit{BLI}) ist die
Entwicklung einer vollständigen Geodaten\-verarbeitungs\--Pipeline, die:

\begin{enumerate}[label=\arabic*.]
      \item räumliche Daten aus \acrshort{osm} und dem Unfallatlas automatisiert in
            eine PostGIS-Datenbank importiert,
      \item für einen beliebigen Standort innerhalb Bremens einen
            \textbf{Livability Score} (0\,--\,100) in Echtzeit berechnet,
      \item den Score in positive und negative Einflussfaktoren aufschlüsselt,
      \item die umliegenden \acrshort{gis}-Features als GeoJSON-Objekte zur
            Visualisierung auf einer interaktiven Karte bereitstellt und
      \item dem Nutzer die Möglichkeit gibt, individuelle Gewichtungen
            (\textit{Präferenzen}) für die einzelnen Faktoren festzulegen.
\end{enumerate}

Das System wird als produktionstaugliche Webanwendung mit Flutter-Frontend und
FastAPI-Backend realisiert. Der vollständige Quellcode ist öffentlich auf
GitHub verfügbar:\\\url{https://github.com/Milad9A/Bremen-Livability-Index}\\Die
Webanwendung ist erreichbar unter:\\\url{https://bremen-livability-frontend.onrender.com}

\section{Abgrenzung}

Die vorliegende Arbeit beschränkt sich räumlich auf das Gebiet der Freien
Hansestadt Bremen (Bounding-Box $53{,}0\text{\textdegree}\,\text{N}$ --
$53{,}2\text{\textdegree}\,\text{N}$,
$8{,}5\text{\textdegree}\,\text{E}$ --
$9{,}0\text{\textdegree}\,\text{E}$). Eine Übertragung auf andere Städte ist
konzeptionell möglich, wird jedoch nicht umgesetzt. Es werden ausschließlich frei
verfügbare, statische Datenquellen verwendet; Echtzeitdaten (z.\,B.\ aktuelle
Lärmwerte, Luftqualität) werden nicht berücksichtigt.

\section{Aufbau der Arbeit}

Kapitel~\ref{ch:grundlagen} führt in die theoretischen Grundlagen der
Geodatenverarbeitung und bestehender Lebensqualitätsindizes ein.
Kapitel~\ref{ch:datenquellen} beschreibt die verwendeten Datenquellen im Detail.
Die Systemarchitektur wird in Kapitel~\ref{ch:architektur} vorgestellt, gefolgt
vom Datenbankdesign (Kapitel~\ref{ch:datenbankdesign}) und der
Bewertungsmethodik (Kapitel~\ref{ch:bewertung}).
Kapitel~\ref{ch:datenerfassung} erläutert die automatisierte Datenerfassung.
Die Implementierung von Backend-\acrshort{api} (Kapitel~\ref{ch:backend})
und Frontend (Kapitel~\ref{ch:frontend}) wird anschließend beschrieben.
Kapitel~\ref{ch:testing} behandelt Testing und Deployment.
Abschließend werden in Kapitel~\ref{ch:ergebnisse} die Ergebnisse diskutiert
und in Kapitel~\ref{ch:fazit} ein Fazit sowie ein Ausblick formuliert.

% !TEX root = ../main.tex
% ============================================================
\chapter{Grundlagen}
\label{ch:grundlagen}
% ============================================================

\section{Lebensqualität und urbane Indizes}

Der Begriff \textit{Lebensqualität} umfasst eine Vielzahl objektiv messbarer und
subjektiv empfundener Dimensionen, darunter Gesundheitsversorgung, Bildung,
Sicherheit, Umweltqualität und infrastrukturelle Erreichbarkeit
\parencite{economist_gla}. Internationale Ansätze wie der \textit{Global
  Liveability Index} (EIU) oder der \textit{Quality of Living Index} (Mercer)
bewerten Städte auf Grundlage makroskopischer Indikatoren
\parencite{mercer2019}. Beide operieren auf Stadtebene und bieten keine
feinräumige Auflösung innerhalb einer Stadt.

Ziel des Bremen Livability Index ist es, einen \textbf{mikroskaligen}
Lebensqualitätsindex zu realisieren, der für \textit{jeden beliebigen Punkt}
innerhalb des Stadtgebiets einen Score berechnet. Dazu werden
\acrfull{gis}-Methoden eingesetzt, um die räumliche Nähe zu positiven und
negativen Infrastrukturmerkmalen quantitativ zu erfassen.

\section{Geoinformationssysteme und räumliche Datenbanken}
\label{sec:gis_grundlagen}

Ein \acrfull{gis} dient der Erfassung, Verwaltung, Analyse und Darstellung
raumbezogener Daten. Im Kontext dieses Projekts werden insbesondere zwei
Fähigkeiten benötigt:

\begin{itemize}
  \item \textbf{Räumliche Abfragen}: Bestimmung aller Objekte innerhalb eines
        definierten Radius um einen Punkt (\textit{proximity queries}).
  \item \textbf{Distanzberechnung}: Berechnung der ellipsoidalen Entfernung
        zwischen geographischen Koordinaten unter Berücksichtigung der
        Erdkrümmung.
\end{itemize}

Die Datenbank PostgreSQL bietet mit der Erweiterung \textbf{PostGIS}
\parencite{postgis_docs} eine leistungsfähige räumliche Datenbanklösung.
PostGIS unterstützt sowohl den Datentyp \texttt{GEOMETRY} (kartesische Ebene)
als auch \texttt{GEOGRAPHY} (ellipsoidale Berechnung auf dem WGS\,84-Ellipsoid).
Für die exakte Entfernungsberechnung in Metern wird in diesem Projekt
ausschließlich der Typ \texttt{GEOGRAPHY} verwendet.

Zentrale PostGIS-Funktionen im Projekt sind
\texttt{ST\_DWithin} (Proximity-Prüfung mit GiST-Index),
\texttt{ST\_Distance} (ellipsoidale Entfernung in Metern),
\texttt{ST\_MakePoint}/\texttt{ST\_SetSRID} (Punkterzeugung mit \acrshort{crs}-Zuweisung)
und \texttt{ST\_AsGeoJSON} (Konvertierung für das Frontend).

\section{Koordinatenreferenzsystem}
\label{sec:crs}

Das gesamte Projekt verwendet durchgängig \textbf{EPSG:4326 (\acrshort{wgs84})}
als einziges \acrlong{crs} -- für die Datenbank, die \acrshort{api}-Ein- und
Ausgabe sowie alle \acrshort{osm}-Quelldaten.

Die Rohdaten des Unfallatlas liegen im Quellformat EPSG:25832
(ETRS89\,/\,\acrshort{utm} Zone\,32N) vor. Sie werden jedoch bereits während
der Datenerfassung (Kapitel~\ref{ch:datenerfassung}) automatisch nach EPSG:4326
reprojiziert, sodass EPSG:25832 ausschließlich als Eingangsformat des
Ingestion-Skripts auftritt und im restlichen System nicht mehr in Erscheinung
tritt.

Die Wahl von EPSG:4326 als Datenbankformat in Kombination mit dem PostGIS-Typ
\texttt{GEOGRAPHY} stellt sicher, dass alle Distanzberechnungen korrekt auf dem
\acrshort{wgs84}-Ellipsoid erfolgen -- ohne die Notwendigkeit einer zusätzlichen
Projektion \parencite{postgis_docs}.

\section{REST-APIs und das Client-Server-Modell}

Die Kommunikation zwischen Frontend und Backend erfolgt über eine
\acrfull{rest}-\acrshort{api}. Das Backend stellt HTTP-Endpunkte bereit, die
JSON-formatierte Anfragen entgegennehmen und Antworten zurückgeben.
Dieses zustandslose Architekturmuster ermöglicht eine klare Trennung von
Darstellungs- und Geschäftslogik sowie eine einfache Skalierbarkeit
\parencite{fastapi_docs}. Die \acrshort{api} ist öffentlich erreichbar unter
\url{https://bremen-livability-backend.onrender.com}; eine interaktive
Swagger-Dokumentation steht unter \texttt{/docs} zur Verfügung.
Abbildung~\ref{fig:client_server} zeigt den Kommunikationsfluss.

\begin{figure}[H]
  \centering
  \begin{tikzpicture}[
    box/.style={draw, rounded corners, minimum width=2.6cm,
                minimum height=1.0cm, align=center, font=\small},
    client/.style={box, fill=NavyBlue!12},
    server/.style={box, fill=OliveGreen!15},
    db/.style={box, fill=Orange!12},
    arr/.style={-{Stealth[length=5pt]}, thick}
  ]
    \node[client] at (0, 0)    (fe) {Flutter App\\(Frontend)};
    \node[server] at (5.5, 0)  (be) {FastAPI\\(Backend)};
    \node[db]     at (11, 0)   (db) {PostGIS\\(Datenbank)};

    \draw[arr] ([yshift= 3pt]fe.east)  -- node[above, font=\scriptsize] {JSON-Request} ([yshift= 3pt]be.west);
    \draw[arr] ([yshift=-3pt]be.west)  -- node[below, font=\scriptsize] {JSON-Response} ([yshift=-3pt]fe.east);
    \draw[arr] ([yshift= 3pt]be.east)  -- node[above, font=\scriptsize] {SQL / ST\_DWithin} ([yshift= 3pt]db.west);
    \draw[arr] ([yshift=-3pt]db.west)  -- node[below, font=\scriptsize] {GeoJSON} ([yshift=-3pt]be.east);
  \end{tikzpicture}
  \caption{Client-Server-Kommunikation über die REST-API}
  \label{fig:client_server}
\end{figure}

\section{Das BLoC-Design-Pattern}
\label{sec:bloc_pattern}

Das \acrfull{bloc}-Design-Pattern \parencite{flutter_bloc} trennt in Flutter-
Anwendungen die UI-Schicht von der Geschäftslogik: Die Benutzeroberfläche sendet
\textit{Events} an den BLoC, der diese verarbeitet und neue \textit{States}
emittiert. Die UI reagiert reaktiv auf Zustandsänderungen. In Kombination mit
dem \texttt{Freezed}-Code\-generator entstehen typsichere, unveränderliche
(\textit{immutable}) Zustandsobjekte, die eine vorhersagbare
Zustandsverwaltung gewährleisten.
Abbildung~\ref{fig:bloc_pattern} veranschaulicht diesen Kreislauf.

\begin{figure}[H]
  \centering
  \begin{tikzpicture}[
    box/.style={draw, rounded corners, minimum width=2.8cm,
                minimum height=1.0cm, align=center, font=\small},
    ui/.style={box, fill=NavyBlue!12},
    bloc/.style={box, fill=OliveGreen!15},
    state/.style={box, fill=Orange!12},
    arr/.style={-{Stealth[length=5pt]}, thick}
  ]
    \node[ui]    at (0, 0)   (uinode) {UI\\(Widget)};
    \node[bloc]  at (4.5, 0) (blocnode) {BLoC\\(Geschäftslogik)};
    \node[state] at (4.5, -2.5) (statenode) {State\\(\texttt{Freezed})};

    \draw[arr] (uinode.east) -- node[above, font=\scriptsize] {Event} (blocnode.west);
    \draw[arr] (blocnode.south) -- node[right, font=\scriptsize] {emittiert} (statenode.north);
    \draw[arr] (statenode.west) -- node[below, font=\scriptsize] {reaktives Rebuild} (uinode.south east);
  \end{tikzpicture}
  \caption{BLoC-Pattern: Event-State-Kreislauf}
  \label{fig:bloc_pattern}
\end{figure}

% !TEX root = ../main.tex
% ============================================================
\chapter{Datenquellen}
\label{ch:datenquellen}
% ============================================================

Für die Berechnung des Livability Scores werden zwei komplementäre offene
Datenquellen herangezogen: \acrfull{osm} für allgemeine Geodaten und der
\textit{Unfallatlas} für verkehrsbezogene Unfalldaten.

\section{OpenStreetMap}
\label{sec:osm}

\acrshort{osm} \parencite{osm_wiki} ist ein kollaboratives
Kartenprojekt, das weltweit freie Geodaten unter der Open Database License
(ODbL~1.0) bereitstellt \parencite{osm_odbl}. Die Daten werden von einer
Community aus über 10 Millionen registrierten Nutzern gepflegt und umfassen
Punkte (\textit{Nodes}), Wege (\textit{Ways}) und Relationen mit semantischen
Tags.

\subsection{Abfrage über die Overpass API}

Der Zugriff auf die \acrshort{osm}-Daten erfolgt über die \textbf{Overpass API}
\parencite{overpass_api}, eine spezialisierte Leseschnittstelle für räumliche
Abfragen. Die Abfragen verwenden die Overpass-QL-Syntax mit einem
Bounding-Box-Filter für das Stadtgebiet Bremen:

\begin{lstlisting}[style=python,caption={Bounding-Box-Definition für Bremen},label={lst:bbox}]
BREMEN_BBOX = {
    "south": 53.0,
    "west":  8.5,
    "north": 53.2,
    "east":  9.0
}
\end{lstlisting}

Dieses Gebiet von ca.\ 420\,km² umfasst das gesamte Stadtgebiet der
Freien Hansestadt Bremen einschließlich Bremerhaven.

\subsection{Datenkategorien}

Aus \acrshort{osm} werden 20 thematische Kategorien extrahiert, die als positive
oder negative Einflussfaktoren in die Bewertung einfließen (Tabelle~\ref{tab:osm_kategorien}).

\begin{table}[H]
      \centering
      \caption{OSM-Datenkategorien und verwendete Tags}
      \label{tab:osm_kategorien}
      \scriptsize
      \begin{tabularx}{\textwidth}{lXll}
            \toprule
            \textbf{Kategorie} & \textbf{OSM-Tags}                                                                 & \textbf{Geom.} & \textbf{Einfl.} \\
            \midrule
            Bäume              & \texttt{natural=tree}                                                             & Point          & +               \\
            Parks              & \texttt{leisure=park}                                                             & Polygon        & +               \\
            Nahversorgung      & \texttt{amenity=supermarket|cafe|restaurant|bank|post\_office|bakery|butcher}     & Point          & +               \\
            ÖPNV               & \texttt{highway=bus\_stop}, \texttt{railway=tram\_stop}                           & Point          & +               \\
            Gesundheit         & \texttt{amenity=hospital|pharmacy|doctors|clinic}                                 & Point          & +               \\
            Fahrrad            & \texttt{highway=cycleway}, \texttt{cycleway=*}, \texttt{amenity=bicycle\_parking} & Pt/Ln          & +               \\
            Bildung            & \texttt{amenity=school|university|college|kindergarten|library}                   & Point          & +               \\
            Sport/Freizeit     & \texttt{leisure=sports\_centre|swimming\_pool|playground|pitch}                   & Point          & +               \\
            Fußgänger          & \texttt{highway=pedestrian|footway}                                               & Line           & +               \\
            Kultur             & \texttt{tourism=museum|gallery}, \texttt{amenity=theatre|cinema}                  & Point          & +               \\
            \addlinespace
            Industrie          & \texttt{landuse=industrial}                                                       & Polygon        & $-$             \\
            Hauptstraßen       & \texttt{highway=motorway|trunk|primary}                                           & Line           & $-$             \\
            Lärm               & \texttt{amenity=nightclub|bar|pub|fast\_food|car\_repair}                         & Point          & $-$             \\
            Eisenbahn          & \texttt{railway=rail}                                                             & Line           & $-$             \\
            Tankstellen        & \texttt{amenity=fuel}                                                             & Point          & $-$             \\
            Abfall             & \texttt{landuse=landfill}, \texttt{amenity=recycling}                             & Pt/Pg          & $-$             \\
            Strom              & \texttt{power=substation|plant|generator}                                         & Pt/Pg          & $-$             \\
            Parkplätze         & \texttt{amenity=parking} (Polygon)                                                & Polygon        & $-$             \\
            Flughäfen          & \texttt{aeroway=aerodrome|helipad}                                                & Pt/Pg          & $-$             \\
            Baustellen         & \texttt{landuse=construction}                                                     & Polygon        & $-$             \\
            \bottomrule
      \end{tabularx}
\end{table}

\subsection{Datenqualität und Vollständigkeit}
\label{sec:osm_datenqualitaet}

Die Qualität der \acrshort{osm}-Daten variiert je nach Kategorie. Für
städtische Gebiete in Deutschland gilt \acrshort{osm} als weitgehend vollständig,
insbesondere bei Straßen, Gebäuden und öffentlichen Einrichtungen. Bei Bäumen
und Fahrradinfrastruktur bestehen dagegen Lücken, da diese Objekte häufig erst
durch spezialisierte Mapping-Kampagnen erfasst werden. Eine systematische
Validierung der Datenvollständigkeit liegt außerhalb des Projektumfangs, wird
jedoch in Kapitel~\ref{ch:ergebnisse} diskutiert.

% ──────────────────────────────────────────────────────────────
\section{Unfallatlas}
\label{sec:unfallatlas}

Der \textit{Unfallatlas} \parencite{unfallatlas} ist ein Angebot der
Statistischen Ämter des Bundes und der Länder und enthält georeferenzierte
Straßenverkehrsunfälle mit Personenschaden ab dem Jahr 2016. Die Daten stehen
als Open Data unter der Lizenz \texttt{dl-de/by-2-0} zum Download bereit
\parencite{unfallatlas_download}.

\subsection{Datenformat und Filterung}

Die Unfalldaten werden als gezippte CSV-Dateien im Koordinatenreferenzsystem
EPSG:25832 (\acrshort{utm} Zone 32N) bereitgestellt. Für das Projekt werden
die Daten nach dem Bundeslandschlüssel \texttt{ULAND=4} (Bremen) gefiltert.

\begin{table}[H]
      \centering
      \caption{Schweregrade der Unfälle im Unfallatlas}
      \label{tab:unfall_severity}
      \begin{tabular}{ccl}
            \toprule
            \textbf{UKATEGORIE} & \textbf{Schweregrad} & \textbf{Beschreibung}       \\
            \midrule
            1                   & \texttt{fatal}       & Unfall mit Getöteten        \\
            2                   & \texttt{severe}      & Unfall mit Schwerverletzten \\
            3                   & \texttt{minor}       & Unfall mit Leichtverletzten \\
            \bottomrule
      \end{tabular}
\end{table}

\subsection{Koordinatentransformation}

Da die Unfalldaten im projizierten System EPSG:25832 vorliegen, die Datenbank
jedoch EPSG:4326 (\acrshort{wgs84}) erwartet, erfolgt bei der Datenerfassung
eine automatische Reprojektion mithilfe der Python-Bibliothek GeoPandas
\parencite{geopandas}. Die Koordinatenspalten
(\path{XGCSWGS84}/\path{YGCSWGS84} bzw.\
\path{LINREFX}/\path{LINREFY}) werden automatisch erkannt.
Zudem wird das deutsche Dezimalkomma-Format
(z.\,B.\ \texttt{53,0793}) in Dezimalpunkt-Format konvertiert.

\subsection{Zeitraum und Umfang}

Es stehen Daten für die Jahre 2016\,--\,2024 zur Verfügung. Standardmäßig wird
das aktuellste verfügbare Jahr (2024) importiert. Die Anzahl der Unfälle in
Bremen variiert dabei je nach Jahr zwischen ca.\ 1.500 und 2.500
Datensätzen.

% !TEX root = ../main.tex
% ============================================================
\chapter{Systemarchitektur}
\label{ch:architektur}
% ============================================================

Der Bremen Livability Index ist als klassische \textbf{Client-Server-Architektur}
mit drei Schichten konzipiert: einer räumlichen Datenbank, einem
\acrshort{rest}-Backend und einem plattformübergreifenden Frontend. Dieses
Kapitel gibt einen Überblick über den Gesamtaufbau und den Technologiestack.

\section{Architekturüberblick}
\label{sec:architektur_ueberblick}

Abbildung~\ref{fig:architektur} zeigt die zentralen Komponenten und
deren Zusammenspiel. Durch die Trennung der Schichten lassen sich Backend und
Frontend unabhängig voneinander entwickeln, testen und deployen.

Die Kernfunktionalität -- räumliche Näheanalysen mittels PostGIS-Funktionen wie
\texttt{ST\_DWithin} und \texttt{GEOGRAPHY}-Datentypen -- erfordert eine
vollwertige räumliche Datenbank, die Firebase/Firestore nicht bieten kann.
Gleichzeitig stellt Firebase Authentication bewährte OAuth- und
Magic-Link-Flows bereit, deren Eigenimplementierung unverhältnismäßig aufwändig
wäre. Die Architektur trennt daher bewusst: \emph{geodatenintensive Logik}
verbleibt im PostGIS-Backend, während \emph{Identitätsverwaltung und
  Favoritensynchronisation} über Firebase-Dienste abgewickelt werden.

\begin{figure}[H]
  \centering
  \begin{tikzpicture}[
    node distance=1.6cm and 1.8cm,
    every node/.style={font=\small},
    box/.style={draw, rounded corners=3pt, minimum width=4.8cm,
                minimum height=0.9cm, align=center, fill=#1!12,
                draw=#1!60, text=#1!80!black},
    box/.default=teal,
    svc/.style={draw, rounded corners=3pt, minimum width=3.4cm,
                minimum height=0.75cm, align=center, fill=#1!10,
                draw=#1!50, text=#1!70!black, font=\small},
    svc/.default=gray,
    arr/.style={-{Stealth[length=5pt]}, thick, color=#1!70},
    arr/.default=teal,
    lbl/.style={font=\scriptsize\sffamily, text=black!60},
  ]
    % ── Main stack ──
    \node[box=teal]  (fe)  {Flutter-App\\[-2pt]
                             \scriptsize Web\,/\,iOS\,/\,Android\,/\,Desktop};
    \node[box=blue,  below=of fe]  (be)  {FastAPI-Backend\\[-2pt]
                             \scriptsize Docker auf Render.com};
    \node[box=violet, below=of be] (db)  {PostgreSQL\,16 + PostGIS\,3.4\\[-2pt]
                             \scriptsize Neon.tech (Serverless)};

    % ── Arrows main ──
    \draw[arr=teal]   (fe) -- node[lbl, right, xshift=2pt] {HTTP\,/\,JSON} (be);
    \draw[arr=blue]   (be) -- node[lbl, right, xshift=2pt] {SQL\,/\,PostGIS} (db);

    % ── Side services (Firebase group, aligned to Flutter row) ──
    \node[svc=orange, right=2.2cm of fe, yshift=0.4cm] (auth) {Firebase Auth\\[-2pt]
                             \scriptsize Authentifizierung};
    \node[svc=orange, below=0.45cm of auth] (fire) {Firestore\\[-2pt]
                             \scriptsize Favoriten\,\&\,Präferenzen};
    % ── Side services (External APIs, aligned to Backend row) ──
    \node[svc=gray,   right=2.2cm of be, yshift=0.4cm] (nom)  {Nominatim API\\[-2pt]
                             \scriptsize Geokodierung};
    \node[svc=gray,   below=0.45cm of nom] (over) {Overpass API\\[-2pt]
                             \scriptsize OSM-Daten (offline)};

    % ── Arrows side ──
    \draw[arr=orange] (fe.east) -- (auth.west);
    \draw[arr=orange] (fe.east) ++(0,-0.15) -| ([xshift=-4pt]fire.west) -- (fire.west);
    \draw[arr=gray]   (be.east) -- (nom.west);
    \draw[arr=gray]   (be.east) ++(0,-0.15) -| ([xshift=-4pt]over.west) -- (over.west);

    % ── Background groups ──
    \begin{scope}[on background layer]
      \node[fit=(auth)(fire), fill=orange!5, draw=orange!25,
            rounded corners=5pt, inner sep=6pt] {};
      \node[fit=(nom)(over), fill=gray!5, draw=gray!25,
            rounded corners=5pt, inner sep=6pt] {};
    \end{scope}
  \end{tikzpicture}
  \caption{Systemarchitektur des Bremen Livability Index}
  \label{fig:architektur}
\end{figure}

\section{Projektstruktur}

Das Projekt ist als öffentliches GitHub-Repository unter \\
\url{https://github.com/Milad9A/Bremen-Livability-Index} verfügbar und in
drei Hauptverzeichnisse gegliedert:

\begin{description}
  \item[\texttt{backend/}] Enthält das FastAPI-Backend mit den Unterordnern
        \texttt{app/} (Endpunkte, Modelle), \texttt{core/} (Datenbank, Scoring,
        Logging), \texttt{services/} (Geokodierung), \texttt{scripts/}
        (Datenerfassung) und \texttt{tests/} (Unit-Tests).
  \item[\texttt{frontend/bli/}] Enthält die Flutter-Anwendung mit der
        Verzeichnisstruktur \texttt{lib/core/} (Services, Theme, Widgets),
        \texttt{lib/features/} (Auth, Map, Favorites, Preferences, Onboarding)
        und plattformspezifischen Ordnern (\texttt{android/}, \texttt{ios/},
        \texttt{web/}, \texttt{macos/}, \texttt{windows/}, \texttt{linux/}).
  \item[\texttt{documentation/}] Enthält die vorliegende LaTeX-Dokumentation
        mit separaten Kapiteldateien und dem Literaturverzeichnis.
\end{description}

\section{Technologiestack}
\label{sec:techstack}

Tabelle~\ref{tab:techstack} gibt einen Überblick über alle eingesetzten
Technologien und begründet deren Auswahl.

\begin{table}[ht!]
  \centering
  \caption{Verwendete Technologien}
  \label{tab:techstack}
  \small
  \begin{tabularx}{\textwidth}{l>{\raggedright}p{4.5cm}>{\raggedright\arraybackslash}X}
    \toprule
    \textbf{Schicht} & \textbf{Technologie}                                                 & \textbf{Begründung} \\
    \midrule
    Datenbank
                     & PostgreSQL~16 + PostGIS~3.4
                     & Leistungsfähigste Open-Source-Lösung für räumliche Daten; native
    \texttt{GEOGRAPHY}-Unterstützung \parencite{postgis_docs}                                                     \\
    \addlinespace
    Backend
                     & FastAPI 0.115 (Python)
                     & Hohe Performance (ASGI), automatische OpenAPI-Doku,
    native Pydantic-Validierung \parencite{fastapi_docs}                                                          \\
    \addlinespace
    Validierung
                     & Pydantic 2.x
                     & Laufzeit-Datenvalidierung über Python-Typannotationen; bildet die
    Grundlage für FastAPI-Request-Validierung und SQLModel-Typisierung                                            \\
    \addlinespace
    \acrshort{orm}
                     & SQLModel + GeoAlchemy2
                     & SQLAlchemy-Leistung mit Pydantic-Typisierung; PostGIS-Funktionen
    als Python-Objekte \parencite{sqlmodel_docs, geoalchemy2_docs}                                                \\
    \addlinespace
    Frontend
                     & Flutter 3.x (Dart)
                     & Plattformübergreifend aus einer Codebasis
    \parencite{flutter_docs}                                                                                      \\
    \addlinespace
    Karte
                     & flutter\_map 8.x
                     & Open-Source-Kartenwidget; CartoDB Voyager Tiles
    \parencite{flutter_map}                                                                                       \\
    \addlinespace
    State Mgmt.
                     & flutter\_bloc 9.x
                     & \acrshort{bloc}-Muster für reaktive, testbare Zustandsverwaltung
    \parencite{flutter_bloc}                                                                                      \\
    \addlinespace
    Auth
                     & Firebase Authentication
                     & Google, GitHub, Magic-Link, anonym; serverseitige Token-Validierung
    \parencite{firebase_docs}                                                                                     \\
    \addlinespace
    Hosting
                     & Render.com, Neon.tech
                     & Containerbasiertes Hosting + Serverless-DB im Free Tier; EU-Standort
    \parencite{render_docs, neontech}                                                                             \\
    \addlinespace
    Container
                     & Docker
                     & Reproduzierbare Build- und Deployment-Umgebung
    \parencite{docker_docs}                                                                                       \\
    \bottomrule
  \end{tabularx}
\end{table}

\section{Deployment-Architektur}
\label{sec:deployment}

Das Deployment erfolgt vollständig cloudbasiert über Infrastruktur im Free Tier
und wird über eine deklarative \texttt{render.yaml}-Konfiguration gesteuert:

\begin{itemize}
  \item \textbf{Backend}: Docker-Container auf Render.com (Web Service,
        Region Frankfurt). Der Container wird aus dem \texttt{backend/Dockerfile}
        gebaut und stellt den FastAPI-Server auf Port~8000 bereit. Die
        Umgebungsvariable \texttt{DATABASE\_URL} verbindet das Backend mit der
        Neon.tech-Datenbank.
  \item \textbf{Frontend (Web)}: Statische Flutter-Webanwendung auf Render.com
        (Static Site). Das Build-Skript \texttt{render\_build.sh} führt
        \texttt{flutter build web} aus. Eine SPA-Rewrite-Regel
        (\texttt{/* $\rightarrow$ /index.html}) stellt clientseitiges Routing
        sicher.
  \item \textbf{Frontend (Mobile \& Desktop)}: Ein GitHub-Actions-Workflow
        (\texttt{build-release.yml}) wird nach jedem erfolgreichen Frontend-Test
        auf dem \texttt{master}-Branch automatisch ausgelöst. Er baut die App
        für Android (APK), Windows (ZIP), Linux (tarball) und macOS (ZIP) und
        veröffentlicht alle vier Artefakte als einheitliches GitHub~Release mit
        Zeitstempel-Tag. Desktop-Builds sind ad-hoc-signiert und erfordern ggf.
        eine Sicherheitsausnahme beim ersten Start.
  \item \textbf{Datenbank}: PostgreSQL~16 mit PostGIS~3.4 auf Neon.tech
        (Serverless). Die Datenbank skaliert automatisch und bietet
        branching-fähige Entwicklungsumgebungen.
  \item \textbf{E-Mail-Redirect}: Firebase Hosting leitet Magic-Link-URLs
        an die Flutter-App weiter.
\end{itemize}

% !TEX root = ../main.tex
% ============================================================
\chapter{Datenbankdesign}
\label{ch:datenbankdesign}
% ============================================================

Die persistente Speicherung und räumliche Abfrage der Geodaten erfolgt über
PostgreSQL~16 mit der Erweiterung PostGIS~3.4
\parencite{postgis_docs, postgresql_docs} mittels erweiterter
\acrfull{sql}-Abfragen. Dieses Kapitel beschreibt das Datenbankschema, die
Tabellenstruktur und die Indexierung.

\section{Schema-Organisation}

Alle projektspezifischen Tabellen befinden sich im dedizierten Schema
\texttt{gis\_data}. Dieses Schema wird bei der erstmaligen Initialisierung
durch das Skript \texttt{init\_db.sql} angelegt, zusammen mit der PostGIS-
Erweiterung:

\begin{lstlisting}[style=sql,caption={Schema- und PostGIS-Initialisierung},label={lst:initdb}]
CREATE EXTENSION IF NOT EXISTS postgis;
CREATE SCHEMA IF NOT EXISTS gis_data;
\end{lstlisting}

\section{Tabellenstruktur}
\label{sec:tabellen}

Das Schema umfasst \textbf{21 Tabellen}, die sich in zwei Gruppen
unterteilen lassen:

\begin{enumerate}
	\item \textbf{Positive Einflussfaktoren} (10 Tabellen):
	      \path{trees}, \path{parks}, \path{amenities},
	      \path{public_transport}, \path{healthcare},
	      \path{bike_infrastructure}, \path{education},
	      \path{sports_leisure}, \path{pedestrian_infrastructure},
	      \path{cultural_venues}
	\item \textbf{Negative Einflussfaktoren} (11 Tabellen):
	      \path{accidents}, \path{industrial_areas},
	      \path{major_roads}, \path{noise_sources},
	      \path{railways}, \path{gas_stations},
	      \path{waste_facilities}, \path{power_infrastructure},
	      \path{parking_lots}, \path{airports},
	      \path{construction_sites}
\end{enumerate}

Jede Geo-Tabelle folgt einem einheitlichen Aufbau:

\begin{table}[H]
	\centering
	\caption{Gemeinsames Spaltenschema der Geo-Tabellen}
	\label{tab:common_schema}
	\begin{tabularx}{\textwidth}{llX}
		\toprule
		\textbf{Spalte}      & \textbf{Datentyp}              & \textbf{Beschreibung}     \\
		\midrule
		\texttt{id}          & \texttt{SERIAL PRIMARY KEY}    & Auto-Inkrement-ID         \\
		\texttt{osm\_id}     & \texttt{BIGINT}                & OpenStreetMap-Objekt-ID
		(entfällt bei Unfalldaten)                                                        \\
		\texttt{name}        & \texttt{TEXT}                  & Bezeichnung (optional)    \\
		\texttt{geometry}    & \texttt{GEOGRAPHY(type, 4326)} & Räumliches Objekt im
		\acrshort{wgs84}-System                                                           \\
		\texttt{created\_at} & \texttt{TIMESTAMP}             & Zeitstempel der Erfassung \\
		\bottomrule
	\end{tabularx}
\end{table}

Einige Tabellen verfügen über zusätzliche Typspalten, z.\,B.\
\texttt{amenity\_type} (Art der Einrichtung), \texttt{severity} (Unfallschweregrad),
\texttt{transport\_type} (Bus/Tram) oder \texttt{healthcare\_type}.

\section{Geometrietypen}

Abhängig von der Art der Geoobjekte werden drei Typen verwendet:
\texttt{POINT} (z.\,B.\ Bäume, Haltestellen, Unfälle),
\texttt{LINESTRING} (Straßen, Schienen, Rad-/Fußwege) und
\texttt{POLYGON} (Parks, Industriegebiete, Parkplätze, Flughäfen).
Alle Geometrien werden als \texttt{GEOGRAPHY} (nicht \texttt{GEOMETRY})
gespeichert, sodass Distanzberechnungen automatisch auf dem
\acrshort{wgs84}-Ellipsoid in Metern erfolgen.

\section{Räumliche Indizierung}
\label{sec:indexierung}

Für jede Geometriespalte wird ein \textbf{GiST-Index} (\textit{Generalized
	Search Tree}) angelegt. GiST-Indizes ermöglichen effiziente räumliche Abfragen,
indem sie die Geometrien in hierarchische Bounding-Boxen partitionieren:

\begin{lstlisting}[style=sql,caption={Beispiel: GiST-Index auf der Tabelle \texttt{trees}},label={lst:gist}]
CREATE INDEX idx_trees_geom
  ON gis_data.trees
  USING GIST (geometry);
\end{lstlisting}

Zusätzlich werden B-Tree-Indizes auf Typspalten (z.\,B.\
\texttt{amenity\_type}, \texttt{transport\_type}) erstellt, um Abfragen mit
Typfiltern zu beschleunigen.

Die Kombination aus GiST-Index und \texttt{ST\_DWithin} ermöglicht es, die
räumlichen Abfragen des Scoring-Algorithmus in wenigen Millisekunden
auszuführen -- selbst bei Tabellen mit über 40.000 Einträgen (z.\,B.\ Bäume).

\section{ORM-Abbildung}

Die Datenbankmodelle werden im Backend durch \textbf{SQLModel}-Klassen
\parencite{sqlmodel_docs} abgebildet, die sowohl als SQLAlchemy-ORM-Modelle
als auch als Pydantic-Validierungsmodelle dienen. Die Geometriespalten
verwenden den Typ \texttt{Geography} aus GeoAlchemy2
\parencite{geoalchemy2_docs}. Alle 21 Geo-Tabellen erben von einer gemeinsamen
Basisklasse \texttt{GISBase} mit der Konfiguration
\texttt{arbitrary\_types\_allowed = True}. Diese Einstellung ist notwendig,
weil Pydantic standardmäßig nur bekannte Python-Typen validiert. Der Typ
\texttt{Geography} aus GeoAlchemy2 ist kein nativer Python-Typ, sondern eine
SQLAlchemy-Spaltendefinition. Ohne diese Einstellung würde Pydantic beim
Initialisieren der Modellklassen mit einem \texttt{RuntimeError} abbrechen.
Der Parameter weist Pydantic an, solche nicht-standardisierten Typen ohne
eigene Validator-Logik zu akzeptieren.

% !TEX root = ../main.tex
% ============================================================
\chapter{Bewertungsmethodik}
\label{ch:bewertung}
% ============================================================

Das Kernstück des Bremen Livability Index ist der Bewertungsalgorithmus, der
für einen gegebenen geographischen Punkt einen \textbf{Livability Score}
zwischen 0 und 100 berechnet.

\section{Bewertungsformel}
\label{sec:formel}

Der Score setzt sich aus einem Basiswert, der Summe positiver Faktoren und
der Summe negativer Faktoren zusammen:

\begin{equation}
  \label{eq:score}
  \text{Score} = \text{clamp}\!\left(
  \underbrace{S_{\text{base}}}_{= 40}
  + \sum_{i=1}^{9} w_i \cdot f_i^{+}
  - \sum_{j=1}^{11} w_j \cdot f_j^{-}
  ,\; 0,\; 100
  \right)
\end{equation}

Dabei ist $S_{\text{base}} = 40$ ein neutraler Ausgangswert,
$f_i^{+}$ bzw.\ $f_j^{-}$ die Einzelscores der positiven/negativen Faktoren
und $w \in \{0{,}0;\; 0{,}5;\; 1{,}0;\; 1{,}5\}$ der nutzerspezifische
Gewichtungsmultiplikator (\textit{ImportanceLevel}: \texttt{excluded},
\texttt{low}, \texttt{medium}, \texttt{high}).

Faktoren mit hohen Zählergebnissen verwenden logarithmische Skalierung
$f(n) = \min(f_{\max},\; \ln(1{+}n) \cdot k)$, die übrigen lineare
Skalierung $f(n) = \min(f_{\max},\; n \cdot k)$.

\section{Positive Faktoren}
\label{sec:positive_faktoren}

\begin{table}[H]
  \centering
  \caption{Positive Einflussfaktoren (Summe Max.: 60)}
  \label{tab:positive}
  \scriptsize
  \begin{tabularx}{\textwidth}{p{2.8cm}ccX}
    \toprule
    \textbf{Faktor} & \textbf{Max.} & \textbf{Radius} & \textbf{Formel}                                                \\
    \midrule
    Grünflächen     & 14            & 175\,m          & $\min(9, \ln(1{+}n_B) \cdot 2{,}0) + \min(5, n_P \cdot 2{,}5)$ \\
    Nahversorgung   & 10            & 550\,m          & $\min(10, \ln(1{+}n) \cdot 2{,}8)$                             \\
    ÖPNV            & 8             & 450\,m          & $\min(8, \ln(1{+}n) \cdot 3{,}5)$                              \\
    Gesundheit      & 6             & 700\,m          & $\min(6, n \cdot 2{,}5)$                                       \\
    Fahrrad         & 6             & 275\,m          & $\min(6, \ln(1{+}n) \cdot 2{,}5)$                              \\
    Bildung         & 5             & 500\,m          & $\min(5, n \cdot 1{,}5)$                                       \\
    Sport/Freizeit  & 4             & 700\,m          & $\min(4, \ln(1{+}n) \cdot 1{,}8)$                              \\
    Kultur          & 4             & 500\,m          & $\min(4, n \cdot 2{,}0)$                                       \\
    Fußgänger       & 3             & 275\,m          & $\min(3, \ln(1{+}n) \cdot 1{,}2)$                              \\
    \bottomrule
  \end{tabularx}
\end{table}

\section{Negative Faktoren}
\label{sec:negative_faktoren}

\begin{table}[H]
  \centering
  \caption{Negative Einflussfaktoren (Summe Max.: 57)}
  \label{tab:negative}
  \scriptsize
  \begin{tabularx}{\textwidth}{p{2.8cm}ccX}
    \toprule
    \textbf{Faktor} & \textbf{Strafe} & \textbf{Radius} & \textbf{Typ}             \\
    \midrule
    Industriegebiet & 10              & 150\,m          & Binär                    \\
    Unfälle         & 8               & 120\,m          & $\min(8, n \cdot 2{,}0)$ \\
    Flughafen       & 7               & 600\,m          & Binär                    \\
    Hauptstraßen    & 6               & 60\,m           & Binär                    \\
    Lärmquellen     & 6               & 75\,m           & $\min(6, n \cdot 2{,}0)$ \\
    Abfall          & 5               & 250\,m          & Binär                    \\
    Eisenbahn       & 5               & 100\,m          & Binär                    \\
    Tankstelle      & 3               & 75\,m           & Binär                    \\
    Strom           & 3               & 75\,m           & Binär                    \\
    Baustelle       & 2               & 125\,m          & Binär                    \\
    Großparkplatz   & 2               & 50\,m           & Binär                    \\
    \bottomrule
  \end{tabularx}
\end{table}

Binäre Faktoren vergeben die volle Strafe, sobald mindestens ein Objekt
im Radius vorhanden ist. Zählerbasierte (Unfälle, Lärm) steigen
proportional, sind aber nach oben begrenzt. Die Suchradien
(50\,--\,700\,m) spiegeln den Einflussbereich der Faktoren wider;
alle Abfragen nutzen \texttt{ST\_DWithin} auf dem \acrshort{wgs84}-Ellipsoid.

\section{Score-Interpretation und Implementierung}

Die App visualisiert den Score durch eine dreistufige Farbcodierung:
$\geq 70$ Grün (gut), $50$--$69$ Orange (mittel), $< 50$ Rot (schlecht).
Der numerische Wert wird direkt angezeigt.

Der Algorithmus ist in \texttt{LivabilityScorer} (\texttt{core/scoring.py})
implementiert. Jeder Faktor wird durch eine eigene statische Methode berechnet;
\texttt{calculate\_score} aggregiert alle Beiträge zu einem
\texttt{LivabilityScoreResponse}-Objekt.

% ============================================================
\chapter{Datenerfassung}
\label{ch:datenerfassung}
% ============================================================

Die Befüllung der Datenbank erfolgt automatisiert über Python-Skripte, die
im Verzeichnis \path{backend/scripts/data_ingestion/} organisiert sind.
Der Einstiegspunkt \texttt{ingest\_all\_data.py} ruft beide
Ingestion-Module nacheinander auf.

\section{OSM-Datenerfassung}
\label{sec:osm_ingestion}

Die Erfassung der 20 \acrshort{osm}-Kategorien erfolgt über die
Python-Bibliothek \texttt{overpy} \parencite{overpy}, einen Wrapper für die
Overpass API \parencite{overpass_api}.

\subsection{Ablauf}

Für jede Kategorie wird folgende Pipeline durchlaufen:

\begin{enumerate}
      \item \textbf{Overpass-Abfrage}: Eine Overpass-QL-Abfrage wird mit dem
            Bounding-Box-Filter für Bremen formuliert und an die Overpass API
            gesendet.
      \item \textbf{Retry-Logik}: Bei Überlastung der API (HTTP~429 oder
            Gateway-Timeout) wird ein exponentieller Backoff mit Jitter
            angewendet:
            \[
                  \text{Wartezeit} = 10 \cdot 2^{\text{Versuch}} + r
                  \qquad r \sim \mathcal{U}(0, 5)
            \]
            Maximal werden 5 Versuche unternommen.
      \item \textbf{Tabelle leeren}: \texttt{TRUNCATE TABLE gis\_data.xxx CASCADE}
            entfernt alle bestehenden Daten, um eine idempotente
            Neuerfassung zu gewährleisten.
      \item \textbf{Bulk-Insert}: Die empfangenen Nodes, Ways oder Relationen
            werden einzeln in die zugehörige Tabelle eingefügt.
\end{enumerate}

\subsection{Geometriekonvertierung}

Die von der Overpass API zurückgegebenen Objekte werden je nach Typ
unterschiedlich verarbeitet:

\begin{description}
      \item[Punkte (Nodes)] werden mit \texttt{ST\_MakePoint(lon, lat)} in einen
            PostGIS-Point konvertiert und mit \texttt{ST\_SetSRID(..., 4326)} dem
            \acrshort{wgs84}-System zugewiesen.
      \item[Polygone (Ways)] Die Koordinaten der Way-Nodes werden zu einem
            WKT-String \texttt{POLYGON(({coords}))} zusammengesetzt. Falls der
            erste und letzte Node nicht identisch sind, wird der Ring automatisch
            geschlossen.
      \item[Linienzüge (Ways)] Analog werden die Koordinaten als
            \texttt{LINESTRING({coords})} zusammengesetzt.
\end{description}

Alle Geometrien werden abschließend nach \texttt{GEOGRAPHY(type, 4326)} gecastet.
Die Datenbankverbindung wird über \texttt{psycopg2} direkt aufgebaut
(ohne ORM), da Bulk-Inserts so effizienter sind.

% ──────────────────────────────────────────────────────────────

\section{Unfallatlas-Datenerfassung}
\label{sec:unfallatlas_ingestion}

Die Erfassung der Unfalldaten \parencite{unfallatlas_download} umfasst
mehrere Schritte, da die Quelldaten in einem anderen Format und
Koordinatensystem vorliegen.

\subsection{Download und Extraktion}

Die Daten werden als gezippte CSV-Datei von der Open-Data-Plattform des
Landes Nordrhein-Westfalen heruntergeladen:

\begin{lstlisting}[basicstyle=\ttfamily\small,breaklines=true,frame=none]
https://www.opengeodata.nrw.de/produkte/transport_verkehr/
  unfallatlas/Unfallorte2024_EPSG25832_CSV.zip
\end{lstlisting}

\subsection{Filterung nach Bremen}

Die CSV-Datei enthält Unfalldaten für ganz Deutschland. Die Filterung auf
Bremen erfolgt anhand des Bundeslandschlüssels:

\begin{lstlisting}[style=python,caption={Filterung auf Bremen},label={lst:filter_bremen}]
# ULAND = 4 entspricht dem Bundesland Bremen
df = df[df["ULAND"] == 4]
\end{lstlisting}

\subsection{Koordinatentransformation}

Die Unfalldaten liegen im projizierten Koordinatensystem EPSG:25832
(\acrshort{utm} Zone 32N) vor \parencite{epsg25832}. Für die Speicherung in
der PostGIS-Datenbank (EPSG:4326) ist eine Reprojektion erforderlich.

Das Skript erkennt automatisch die Koordinatenspalten
(\path{XGCSWGS84}/\path{YGCSWGS84}
bzw.\ \path{LINREFX}/\path{LINREFY}),
konvertiert das deutsche Dezimalkomma-Format und reprojiziert die Daten
mithilfe von GeoPandas \parencite{geopandas} nach EPSG:4326.

\subsection{Schweregrad-Mapping}

Das Feld \texttt{UKATEGORIE} wird in lesbare Schweregrade konvertiert
(siehe Tabelle~\ref{tab:unfall_severity} in Kapitel~\ref{ch:datenquellen}):
$1 \rightarrow \texttt{fatal}$, $2 \rightarrow \texttt{severe}$,
$3 \rightarrow \texttt{minor}$.

\subsection{Verfügbare Jahrgänge}

Es stehen Daten für die Jahrgänge 2016\,--\,2024 zur Verfügung. Das
Ingestion-Skript akzeptiert das gewünschte Jahr als Parameter und versucht,
bei Nicht-Verfügbarkeit automatisch das nächstältere Jahr zu verwenden.

% !TEX root = ../main.tex
% ============================================================
\chapter{Backend-API}
\label{ch:backend}
% ============================================================

Das Backend ist als \acrshort{rest}-konforme \acrshort{api} mit dem
Python-Framework FastAPI \parencite{fastapi_docs} implementiert. Es stellt
die zentrale Schnittstelle zwischen Frontend und Datenbank dar.
Alle Endpunkte können über die interaktive Swagger-Dokumentation unter
\url{https://bremen-livability-backend.onrender.com/docs} erkundet werden.
Für manuelle Tests steht im Repository unter
\texttt{backend/Bremen\_Livability\_Index.postman\_collection.json} eine
Postman-Collection bereit, die alle Endpunkte mit Beispielanfragen abdeckt.

\section{Endpunkte}
\label{sec:endpunkte}

Tabelle~\ref{tab:endpoints} zeigt alle verfügbaren HTTP-Endpunkte der
\acrshort{api}.

\begin{table}[H]
  \centering
  \caption{API-Endpunkte}
  \label{tab:endpoints}
  \small
  \begin{tabularx}{\textwidth}{llX}
    \toprule
    \textbf{Methode} & \textbf{Pfad}                                                          & \textbf{Beschreibung} \\
    \midrule
    GET              & \texttt{/}
                     & API-Metadaten (Version, Endpunktliste)                                                         \\
    GET              & \texttt{/health}
                     & Datenbank-Konnektivitätsprüfung                                                                \\
    POST             & \texttt{/analyze}
                     & \textbf{Kern-Endpunkt}: Berechnung des Livability Scores
    für Koordinaten mit optionalen Präferenzen                                                                        \\
    POST             & \texttt{/geocode}
                     & Adresssuche über Nominatim \parencite{nominatim}                                               \\
    GET              & \texttt{/preferences/defaults}
                     & Gibt Standardpräferenzen, Multiplikatoren \newline und Faktoren zurück                         \\
    POST             & \texttt{/users}
                     & Benutzer anlegen oder aktualisieren                                                            \\
    GET              & \texttt{/users/\{user\_id\}/favorites}
                     & Favoriten eines Benutzers abrufen                                                              \\
    POST             & \texttt{/users/\{user\_id\}/favorites}
                     & Favorit hinzufügen                                                                             \\
    DELETE           & \texttt{/users/\{user\_id\}/favorites/\{fav\_id\}}
                     & Favorit löschen                                                                                \\
    \bottomrule
  \end{tabularx}
\end{table}

Die Favoriten-Endpunkte bilden vollständige \acrfull{crud}-Operationen ab:
Anlegen (POST), Abrufen (GET) und Löschen (DELETE) von gespeicherten
Standorten.

\section{Der \texttt{/analyze}-Endpunkt}
\label{sec:analyze}

Der zentrale Endpunkt nimmt eine Anfrage vom Typ
\texttt{LocationRequest} entgegen und gibt eine Antwort vom Typ
\texttt{LivabilityScoreResponse} zurück. Der Ablauf
umfasst die folgenden Schritte:

\begin{enumerate}
  \item \textbf{Validierung}: Pydantic validiert die Eingabekoordinaten
        ($-90 \leq \text{lat} \leq 90$, $-180 \leq \text{lon} \leq 180$).
  \item \textbf{Räumliche Abfragen}: Für jeden nicht-ausgeschlossenen Faktor
        wird eine PostGIS-\texttt{ST\_DWithin}-Abfrage ausgeführt, die alle
        Objekte innerhalb des faktorspezifischen Radius ermittelt.
  \item \textbf{Score-Berechnung}: Der \texttt{LivabilityScorer} berechnet
        den Einzelscore jedes Faktors und aggregiert den Gesamtscore
        (siehe Kapitel~\ref{ch:bewertung}).
  \item \textbf{GeoJSON-Erzeugung}: Die nahen Objekte werden mit
        \texttt{ST\_AsGeoJSON} in GeoJSON konvertiert und in der
        Antwort als \texttt{nearby\_features} zurückgegeben.
  \item \textbf{Zusammenfassung}: Ein menschenlesbarer Zusammenfassungstext
        wird generiert.
\end{enumerate}

\subsection{Request-Modell}

\begin{lstlisting}[style=json,caption={Beispiel-Request an \texttt{/analyze}},label={lst:analyze_req}]
{
  "latitude": 53.0793,
  "longitude": 8.8017,
  "preferences": {
    "greenery": "high",
    "airport": "excluded",
    "noise": "low"
  }
}
\end{lstlisting}

\subsection{Response-Modell}

Die Antwort enthält den \texttt{score} (0--100), den \texttt{base\_score}
(40.0), die angefragten Koordinaten, eine \texttt{factors}-Liste mit
Aufschlüsselung aller Einzelfaktoren, \texttt{nearby\_features} als
GeoJSON-Objekte gruppiert nach Kategorie sowie eine textuelle
\texttt{summary}.

\begin{lstlisting}[style=json,caption={Beispiel-Response von \texttt{/analyze}},label={lst:analyze_res}]
{
  "score": 72.5,
  "base_score": 40.0,
  "location": { "latitude": 53.0793, "longitude": 8.8017 },
  "factors": [
    { "factor": "greenery",        "value": 18.0, "impact": "positive",
      "description": "Parks and trees nearby" },
    { "factor": "public_transport","value": 7.0,  "impact": "positive",
      "description": "Transit stops nearby" },
    { "factor": "major_roads",     "value": -6.0, "impact": "negative",
      "description": "Major road within 60m" }
  ],
  "nearby_features": {
    "parks": [
      {
        "id": 42,
        "name": "Buergerpark",
        "type": "park",
        "subtype": null,
        "distance": 134.2,
        "geometry": { "type": "Point",
                      "coordinates": [8.8021, 53.0797] }
      }
    ]
  },
  "summary": "Excellent amenities. Good transit access."
}
\end{lstlisting}

\section{Dependency Injection und CORS}
\label{sec:di_cors}

FastAPI nutzt Dependency Injection für Datenbankverbindungen: Die Funktion
\texttt{get\_session()} liefert eine SQLModel-\texttt{Session} als Generator,
sodass jede Anfrage eine eigene Session erhält.
Da Frontend und Backend auf unterschiedlichen Domains gehostet werden,
wird eine \acrfull{cors}-Middleware konfiguriert, deren erlaubte Origins
über die Umgebungsvariable \texttt{CORS\_ORIGINS} gesteuert werden.

\section{Geokodierung}

Der \texttt{/geocode}-Endpunkt delegiert die Adresssuche an den
\texttt{GeocodeService}, der intern die Nominatim-\acrshort{api}
\parencite{nominatim} anspricht. Nominatim ist ein freier Geokodierungsdienst,
der \acrshort{osm}-Daten nutzt und keine API-Schlüssel erfordert.
Die Ergebnisse enthalten Koordinaten, eine formatierte Adresse,
einen Typ und einen Relevanzwert (\texttt{importance}).

% ============================================================
\chapter{Frontend}
\label{ch:frontend}
% ============================================================

Das Frontend ist als plattformübergreifende Anwendung mit dem
UI-Framework Flutter \parencite{flutter_docs} implementiert und unterstützt
Web, iOS, Android, macOS, Windows und Linux.

\section{Architektur und State Management}
\label{sec:frontend_architektur}

Die Anwendung folgt dem \acrfull{bloc}-Entwurfsmuster
(vgl.\ Abschnitt~\ref{sec:bloc_pattern}), das die Geschäftslogik
vollständig von der Darstellungsschicht trennt. Die Architektur gliedert
sich in die folgenden Module:

\begin{description}
      \item[\texttt{lib/core/}] Querschnittskomponenten:
            \path{ApiService} (HTTP-Client via Dio),
            \path{DeepLinkService}, Theme-Definitionen
            (\path{AppTheme}, \path{AppColors},
            \path{AppTextStyles}) und wiederverwendbare Widgets.
      \item[\texttt{lib/features/auth/}] Authentifizierungslogik:
            \texttt{AuthBloc}, \texttt{AuthService}, Login-Screens.
      \item[\texttt{lib/features/map/}] Kern-Feature:
            \texttt{MapBloc} (Karteninteraktionen), \texttt{MapScreen},
            \texttt{ScoreCardView}, \texttt{FloatingSearchBar}.
      \item[\texttt{lib/features/favorites/}] Favoritenverwaltung:
            \texttt{FavoritesBloc}, Datenmodelle.
      \item[\texttt{lib/features/preferences/}] Nutzerpräferenzen:
            \texttt{PreferencesBloc}, \texttt{PreferencesScreen},
            \texttt{PreferencesService}.
      \item[\texttt{lib/features/onboarding/}] Startbildschirm.
\end{description}

Alle Events und States werden mithilfe des Code-Generators \texttt{Freezed}
als \textit{Sealed Unions} definiert. Dies erzwingt typsichere, vollständige
Pattern-Matching-Behandlung in der UI und verhindert undefinierte Zustände.

\section{Kartenansicht}
\label{sec:kartenansicht}

Die Kartenansicht bildet die zentrale Benutzeroberfläche. Sie basiert auf dem
Widget \texttt{FlutterMap} \parencite{flutter_map} mit \textbf{CartoDB Voyager}
Tiles -- einem schlichten Kartenstil ohne störende POI-Beschriftungen.

\begin{itemize}
      \item \textbf{Startzentrum}: $53{,}0793^{\circ}\,\text{N}$, $8{,}8017^{\circ}\,\text{E}$
            (Bremen Innenstadt)
      \item \textbf{Startzoom}: 13
      \item \textbf{Interaktion}: Ein Tipp auf die Karte löst ein
            \texttt{MapTapped}-Event aus, das den \texttt{MapBloc} veranlasst,
            den \texttt{ApiService} anzufragen. Der berechnete Score und die
            nahen Features werden anschließend als Marker und Score-Karte
            dargestellt.
\end{itemize}

Jede Feature-Kategorie erhält einen eigenen Marker mit einer
standardisierten Farbpalette aus \texttt{AppColors}, sodass der Nutzer
die Art der nahen Einrichtungen visuell unterscheiden kann.

\section{Benutzeroberfläche -- Liquid Glass Design}
\label{sec:ui_design}

Das UI-Design folgt einem \textit{Liquid Glass}-Konzept: Die Karte nimmt
den gesamten Bildschirm ein, während alle interaktiven Elemente
(Suchleiste, Score-Karte, Präferenzen) als halbtransparente, gefrostete
Glasflächen über der Karte schweben. Optische Effekte wie leichte
Vergrößerung, Verzerrung und haptisches Feedback auf Mobilgeräten
runden das Interaktionserlebnis ab.

\section{Authentifizierung}
\label{sec:auth}

Die Authentifizierung wird über Firebase Authentication \parencite{firebase_docs}
abgewickelt und unterstützt mehrere Anmeldemethoden:

\begin{table}[H]
      \centering
      \caption{Unterstützte Anmeldemethoden nach Plattform}
      \label{tab:auth_methods}
      \begin{tabularx}{\textwidth}{lX}
            \toprule
            \textbf{Plattform} & \textbf{Anmeldemethoden}                       \\
            \midrule
            Web, iOS, Android
                               & Google, GitHub, E-Mail-Magic-Link, Anonym      \\
            macOS, Windows, Linux
                               & Nur als Gast (Firebase Auth wird übersprungen) \\
            \bottomrule
      \end{tabularx}
\end{table}

Auf Desktop-Plattformen wird die Firebase-Authentifizierung vollständig
umgangen, da macOS-Keychains durch Sandboxing eingeschränkt sind und
Windows/Linux keine nativen OAuth-Desktop-Flows unterstützen. Stattdessen
wird ein lokales \texttt{AppUser.guest()}-Objekt ohne Firebase-Interaktion
erzeugt.

\subsection{Cross-Device E-Mail-Link-Flow}

Wenn ein Nutzer einen Magic Link auf einem anderen Gerät öffnet als dem,
auf dem er die Anmeldung initiiert hat, ist die E-Mail-Adresse nicht
im lokalen Speicher hinterlegt. In diesem Fall erkennt der
\texttt{DeepLinkService} den \texttt{oobCode}-Parameter in der URL,
und der \texttt{AuthBloc} navigiert den Nutzer zu einem
\texttt{EmailLinkPromptScreen}, auf dem er seine E-Mail-Adresse erneut
eingeben kann.

\section{Favoritenverwaltung}
\label{sec:favorites}

Nutzer können analysierte Standorte als Favoriten speichern. Die
Datenhaltung ist zweistufig:

\begin{itemize}
      \item \textbf{Angemeldete Nutzer}: Synchronisation über Firebase Firestore
            und das Backend (\path{/users/{user_id}/favorites}).
      \item \textbf{Gäste}: Lokale Speicherung über \texttt{SharedPreferences}
            (Key-Value-Store des Geräts).
\end{itemize}

Der \texttt{FavoritesBloc} abstrahiert die Speicherstrategie und stellt
eine einheitliche Schnittstelle für das UI bereit.

\section{Adresssuche}

Die \texttt{FloatingSearchBar} ermöglicht die Suche nach Adressen.
Eingaben werden über den \texttt{ApiService} an den
\texttt{/geocode}-Endpunkt des Backends weitergeleitet, der wiederum die
Nominatim-\acrshort{api} abfragt. Die Suchergebnisse werden als
Drop-down-Liste angezeigt; bei Auswahl wird die Karte zur
entsprechenden Koordinate navigiert und automatisch eine
Score-Berechnung ausgelöst.

% ============================================================
\chapter{Testing und Deployment}
\label{ch:testing}
% ============================================================

Dieses Kapitel beschreibt die Qualitätssicherung durch automatisierte Tests
sowie die \acrfull{ci}-Pipeline und die Deployment-Strategie.

\section{Backend-Tests}
\label{sec:backend_tests}

Das Backend wird mit \textbf{pytest} \parencite{pytest_docs} getestet. Die
Testdateien befinden sich im Verzeichnis \texttt{backend/tests/} und decken
vier Bereiche ab:

\begin{table}[H]
      \centering
      \caption{Backend-Testdateien und Testumfang}
      \label{tab:backend_tests}
      \begin{tabularx}{\textwidth}{lrX}
            \toprule
            \textbf{Datei} & \textbf{Tests}                                                & \textbf{Schwerpunkt} \\
            \midrule
            \texttt{test\_scoring.py}
                           & 58
                           & Alle Scoring-Funktionen: Grenzwerte, Rand\-fälle (0, 1, viele
            Objekte), logarithmische Skalierung, binäre Faktoren,
            Importance-Multiplikatoren                                                                            \\
            \texttt{test\_api.py}
                           & --
                           & FastAPI-Endpunkt-Tests mit \texttt{TestClient}                                       \\
            \texttt{test\_database.py}
                           & --
                           & Datenbank-Verbindungstests                                                           \\
            \texttt{test\_main.py}
                           & --
                           & Integrationstests der Hauptanwendung                                                 \\
            \bottomrule
      \end{tabularx}
\end{table}

Die Scoring-Tests sind besonders umfangreich, da der Bewertungsalgorithmus
das Kernstück der Anwendung bildet. Jede der 20 Berechnungsfunktionen wird
mit mindestens den folgenden Szenarien getestet:

\begin{itemize}
      \item \textbf{Leereingabe}: $n = 0$ muss Score $0{,}0$ ergeben.
      \item \textbf{Einzelner Treffer}: Korrektheit der Formeln bei $n = 1$.
      \item \textbf{Sättigungsfall}: Sehr hohe $n$-Werte dürfen das jeweilige
            Maximum nicht überschreiten.
      \item \textbf{Binäre Faktoren}: \texttt{True}/\texttt{False} muss exakt
            die definierte Strafe bzw.\ $0{,}0$ ergeben.
\end{itemize}

\subsection{Code Coverage}

Die Testabdeckung wird mit \texttt{pytest-cov} gemessen und an
Codecov \parencite{codecov} übermittelt. Das Projekt erreicht eine
Abdeckung von über 90\,\%.

\section{Frontend-Tests}
\label{sec:frontend_tests}

Die Flutter-Tests befinden sich in \texttt{frontend/bli/test/} und verwenden
die Bibliotheken \texttt{bloc\_test}, \texttt{mockito} und \texttt{mocktail}
für BLoC-Tests mit gemockten Abhängigkeiten. Getestet werden insbesondere:

\begin{itemize}
      \item \textbf{BLoC-Logik}: Korrekte Zustandsübergänge bei
            Events (z.\,B.\ \path{MapTapped} $\rightarrow$
            \path{MapLoading} $\rightarrow$ \path{MapLoaded})
      \item \textbf{Fehlerbehandlung}: Netzwerkfehler und ungültige
            Serverantworten
      \item \textbf{Authentifizierung}: Login-Flows für verschiedene Provider
\end{itemize}

\section{Continuous Integration}
\label{sec:ci}

Die \acrshort{ci}-Pipeline basiert auf \textbf{GitHub Actions}
\parencite{github_actions} und umfasst drei Workflows:

\begin{table}[H]
      \centering
      \caption{GitHub-Actions-Workflows}
      \label{tab:workflows}
      \begin{tabularx}{\textwidth}{lX}
            \toprule
            \textbf{Workflow} & \textbf{Beschreibung}                                           \\
            \midrule
            \texttt{backend-tests.yml}
                              & Führt \texttt{pytest} bei jedem Push auf das Backend aus;
            übermittelt Coverage an Codecov                                                     \\
            \texttt{frontend-tests.yml}
                              & Führt \texttt{flutter test} bei jedem Push auf das Frontend aus \\
            \texttt{build-release.yml}
                              & Baut APK- (Android), Windows-, macOS- und Linux-Binaries und
            erstellt automatisch GitHub Releases                                                \\
            \bottomrule
      \end{tabularx}
\end{table}

Durch die Aufteilung in drei separate Workflows können Backend- und
Frontend-Tests parallel und unabhängig voneinander ausgeführt werden.

\section{Deployment}
\label{sec:deployment_detail}

Das Deployment erfolgt über die deklarative Datei \texttt{render.yaml} auf
Render.com \parencite{render_docs}. Bei jedem Merge auf den
\texttt{master}-Branch werden Backend und Frontend automatisch neu
deployt.

\begin{itemize}
      \item \textbf{Backend}: Render.com baut den Docker-Container aus
            \texttt{backend/Dockerfile} und startet Uvicorn auf Port~8000.
            Die \texttt{DATABASE\_URL} wird als Secret konfiguriert.
      \item \textbf{Frontend}: Das Build-Skript \texttt{render\_build.sh}
            führt \texttt{flutter build web} aus; das Ergebnis wird als
            statische Website gehostet (SPA-Rewrite auf \texttt{/index.html}).
      \item \textbf{Datenbank}: PostgreSQL~16 mit PostGIS~3.4 auf
            Neon.tech \parencite{neontech} (Serverless, EU-Standort Frankfurt,
            automatische Skalierung auf null bei Inaktivität).
\end{itemize}

% !TEX root = ../main.tex
% ============================================================
\chapter{Ergebnisse und Diskussion}
\label{ch:ergebnisse}
% ============================================================

Dieses Kapitel präsentiert exemplarische Ergebnisse des Bremen Livability
Index und diskutiert Stärken, Schwächen sowie Limitierungen der gewählten
Methodik.

\section{Exemplarische Standortbewertungen}
\label{sec:beispiel_scores}

Um die Funktionalität des Systems zu demonstrieren, wurden Livability Scores
für fünf repräsentative Bremer Standorte mit den Standardpräferenzen
(\texttt{medium} für alle Faktoren) berechnet:

\begin{table}[H]
      \centering
      \caption{Exemplarische Livability Scores für Bremer Standorte}
      \label{tab:beispiel_scores}
      \begin{tabularx}{\textwidth}{Xccl}
            \toprule
            \textbf{Standort} & \textbf{Lat.} & \textbf{Lon.} & \textbf{Score}     \\
            \midrule
            Marktplatz (Innenstadt)
                              & 53.076        & 8.808         & $\sim$\,75\,--\,85 \\
            Bürgerpark (Schwachhausen)
                              & 53.092        & 8.822         & $\sim$\,70\,--\,80 \\
            Universität Bremen
                              & 53.106        & 8.853         & $\sim$\,65\,--\,75 \\
            Industriehafen
                              & 53.120        & 8.760         & $\sim$\,25\,--\,35 \\
            Flughafen Bremen
                              & 53.047        & 8.787         & $\sim$\,20\,--\,30 \\
            \bottomrule
      \end{tabularx}
\end{table}

\textit{Hinweis: Die exakten Scores variieren je nach aktuellem Datenbestand
      in der Datenbank und dem gewählten Abfragepunkt.}

\subsection{Interpretation}

Die Ergebnisse zeigen ein plausibles Muster:

\begin{itemize}
      \item \textbf{Innenstadt und Parks}: Hohe Scores aufgrund dichter
            Nahversorgung, guter ÖPNV-Anbindung und vieler Grünflächen.
      \item \textbf{Universität}: Guter Score durch Bildungseinrichtungen
            und ÖPNV, leicht geringer wegen weniger Nahversorgung.
      \item \textbf{Industriehafen}: Niedriger Score durch starke negative
            Faktoren (Industriegebiete, Hauptstraßen, Eisenbahn) bei
            gleichzeitig geringer positiver Infrastruktur.
      \item \textbf{Flughafen}: Niedrigster Score durch die Kombination
            aus Flughafen-Nähe (7 Strafpunkte), Hauptstraßen und fehlender
            Wohninfrastruktur.
\end{itemize}

\section{Stärken des Systems}

Das System bietet feinräumige Auflösung (Scores für jeden Punkt statt ganzer
Städte), transparente Faktoraufschlüsselung, Personalisierbarkeit durch
dynamische Gewichtung, Echtzeitberechnung dank GiST-Indizes,
ausschließliche Nutzung offener Daten und plattformübergreifende Verfügbarkeit.

\section{Limitierungen}

Das System ist räumlich auf Bremen beschränkt. Die Vollständigkeit der
\acrshort{osm}-Daten variiert je nach Kategorie (vgl.\
Abschnitt~\ref{sec:osm_datenqualitaet}). Echtzeitdaten (Luftqualität,
Lärmpegel) fließen nicht ein. Die Basisgewichtungen basieren auf
Erfahrungswerten statt empirischen Studien. Binäre Faktoren differenzieren
nicht nach Größe/Intensität. Unfalldaten werden nur für ein Jahr importiert.

\section{Verbesserungspotenzial}
\label{sec:verbesserungen}

Auf Basis der identifizierten Limitierungen lassen sich mehrere
Weiterentwicklungen ableiten:

\begin{itemize}
      \item \textbf{Skalierung}: Generalisierung der Bounding-Box-Konfiguration
            und Erweiterung auf weitere deutsche Städte oder den gesamten DACH-Raum.
      \item \textbf{Echtzeitdaten}: Integration von Echtzeit-APIs für
            Luftqualität (z.\,B.\ Umweltbundesamt), Lärmkarten und
            ÖPNV-Verspätungen.
      \item \textbf{Machine Learning}: Erlernung optimaler Gewichtungen aus
            Nutzerfeedback (implizit durch Favoriten, explizit durch Bewertungen).
      \item \textbf{Differenzierte negative Faktoren}: Abstufung der Strafpunkte
            basierend auf der Distanz zum negativen Objekt (\textit{distance
                  decay}).
      \item \textbf{Aggregierte Unfalldaten}: Import und Mittelung über
            mehrere Jahrgänge zur Glättung von Ausreißern.
      \item \textbf{Offene API}: Bereitstellung einer öffentlichen API für
            Drittanwendungen und Forschungszwecke.
\end{itemize}

% !TEX root = ../main.tex
% ============================================================
\chapter{Fazit und Ausblick}
\label{ch:fazit}
% ============================================================

\section{Zusammenfassung}

Im Rahmen des Moduls \textit{Geodatenverarbeitung} an der Hochschule Bremen
wurde mit dem \textbf{Bremen Livability Index} ein vollständiges
Geoinformationssystem konzipiert, implementiert und als produktionstaugliche
Webanwendung bereitgestellt. Das System erfüllt alle fünf in
Kapitel~\ref{ch:einleitung} definierten Ziele: automatisierte Erfassung
von über 60.000 Geodaten-Objekten, Echtzeit-Scoring (0\,--\,100) für
beliebige Standorte, transparente Faktoraufschlüsselung mit
GeoJSON-Visualisierung, interaktive Kartenanwendung und individuelle
Gewichtung durch den Nutzer.

\section{Beantwortung der Zielsetzung}

Die zentrale Fragestellung -- ob sich aus frei verfügbaren Geodaten ein
aussagekräftiger, feinräumiger Lebensqualitätsindex für Bremen ableiten
lässt -- kann positiv beantwortet werden. Die exemplarischen Ergebnisse
(Kapitel~\ref{ch:ergebnisse}) zeigen plausible Unterschiede zwischen
verschiedenen Standorttypen.

\section{Ausblick}

Vielversprechende Weiterentwicklungen umfassen die Skalierung auf weitere
Städte, die Integration von Echtzeitdaten (Luftqualität, Lärmkarten) und
die empirische Validierung der Gewichtungen durch Nutzerbefragungen
oder maschinelles Lernen.


% ==============================================================
%  LITERATURVERZEICHNIS
% ==============================================================
\clearpage
\printbibliography[title={Literaturverzeichnis}]

\end{document}
