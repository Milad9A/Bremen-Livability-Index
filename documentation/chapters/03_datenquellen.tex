% !TEX root = ../main.tex
% ============================================================
\chapter{Datenquellen}
\label{ch:datenquellen}
% ============================================================

Für die Berechnung des Livability Scores werden zwei komplementäre offene
Datenquellen herangezogen: \acrfull{osm} für allgemeine Geodaten und der
\textit{Unfallatlas} für verkehrsbezogene Unfalldaten.

\section{OpenStreetMap}
\label{sec:osm}

\acrshort{osm} \parencite{osm_wiki} ist ein kollaboratives
Kartenprojekt, das weltweit freie Geodaten unter der Open Database License
(ODbL~1.0) bereitstellt \parencite{osm_odbl}. Die Daten werden von einer
Community aus über 10 Millionen registrierten Nutzern gepflegt und umfassen
Punkte (\textit{Nodes}), Wege (\textit{Ways}) und Relationen mit semantischen
Tags.

\subsection{Abfrage über die Overpass API}

Der Zugriff auf die \acrshort{osm}-Daten erfolgt über die \textbf{Overpass API}
\parencite{overpass_api}, eine spezialisierte Leseschnittstelle für räumliche
Abfragen. Die Abfragen verwenden die Overpass-QL-Syntax mit einem
Bounding-Box-Filter für das Stadtgebiet Bremen:

\begin{lstlisting}[style=python,caption={Bounding-Box-Definition für Bremen},label={lst:bbox}]
BREMEN_BBOX = {
    "south": 53.0,
    "west":  8.5,
    "north": 53.2,
    "east":  9.0
}
\end{lstlisting}

Dieses Gebiet von ca.\ 420\,km² umfasst das gesamte Stadtgebiet der
Freien Hansestadt Bremen einschließlich Bremerhaven.

\subsection{Datenkategorien}

Aus \acrshort{osm} werden 20 thematische Kategorien extrahiert, die als positive
oder negative Einflussfaktoren in die Bewertung einfließen (Tabelle~\ref{tab:osm_kategorien}).

\begin{table}[H]
      \centering
      \caption{OSM-Datenkategorien und verwendete Tags}
      \label{tab:osm_kategorien}
      \scriptsize
      \begin{tabularx}{\textwidth}{lXll}
            \toprule
            \textbf{Kategorie} & \textbf{OSM-Tags}                                                                 & \textbf{Geom.} & \textbf{Einfl.} \\
            \midrule
            Bäume              & \texttt{natural=tree}                                                             & Point          & +               \\
            Parks              & \texttt{leisure=park}                                                             & Polygon        & +               \\
            Nahversorgung      & \texttt{amenity=supermarket|cafe|restaurant|bank|post\_office|bakery|butcher}     & Point          & +               \\
            ÖPNV               & \texttt{highway=bus\_stop}, \texttt{railway=tram\_stop}                           & Point          & +               \\
            Gesundheit         & \texttt{amenity=hospital|pharmacy|doctors|clinic}                                 & Point          & +               \\
            Fahrrad            & \texttt{highway=cycleway}, \texttt{cycleway=*}, \texttt{amenity=bicycle\_parking} & Pt/Ln          & +               \\
            Bildung            & \texttt{amenity=school|university|college|kindergarten|library}                   & Point          & +               \\
            Sport/Freizeit     & \texttt{leisure=sports\_centre|swimming\_pool|playground|pitch}                   & Point          & +               \\
            Fußgänger          & \texttt{highway=pedestrian|footway}                                               & Line           & +               \\
            Kultur             & \texttt{tourism=museum|gallery}, \texttt{amenity=theatre|cinema}                  & Point          & +               \\
            \addlinespace
            Industrie          & \texttt{landuse=industrial}                                                       & Polygon        & $-$             \\
            Hauptstraßen       & \texttt{highway=motorway|trunk|primary}                                           & Line           & $-$             \\
            Lärm               & \texttt{amenity=nightclub|bar|pub|fast\_food|car\_repair}                         & Point          & $-$             \\
            Eisenbahn          & \texttt{railway=rail}                                                             & Line           & $-$             \\
            Tankstellen        & \texttt{amenity=fuel}                                                             & Point          & $-$             \\
            Abfall             & \texttt{landuse=landfill}, \texttt{amenity=recycling}                             & Pt/Pg          & $-$             \\
            Strom              & \texttt{power=substation|plant|generator}                                         & Pt/Pg          & $-$             \\
            Parkplätze         & \texttt{amenity=parking} (Polygon)                                                & Polygon        & $-$             \\
            Flughäfen          & \texttt{aeroway=aerodrome|helipad}                                                & Pt/Pg          & $-$             \\
            Baustellen         & \texttt{landuse=construction}                                                     & Polygon        & $-$             \\
            \bottomrule
      \end{tabularx}
\end{table}

\subsection{Datenqualität und Vollständigkeit}
\label{sec:osm_datenqualitaet}

Die Qualität der \acrshort{osm}-Daten variiert je nach Kategorie. Für
städtische Gebiete in Deutschland gilt \acrshort{osm} als weitgehend vollständig,
insbesondere bei Straßen, Gebäuden und öffentlichen Einrichtungen. Bei Bäumen
und Fahrradinfrastruktur bestehen dagegen Lücken, da diese Objekte häufig erst
durch spezialisierte Mapping-Kampagnen erfasst werden. Eine systematische
Validierung der Datenvollständigkeit liegt außerhalb des Projektumfangs, wird
jedoch in Kapitel~\ref{ch:ergebnisse} diskutiert.

% ──────────────────────────────────────────────────────────────
\section{Unfallatlas}
\label{sec:unfallatlas}

Der \textit{Unfallatlas} \parencite{unfallatlas} ist ein Angebot der
Statistischen Ämter des Bundes und der Länder und enthält georeferenzierte
Straßenverkehrsunfälle mit Personenschaden ab dem Jahr 2016. Die Daten stehen
als Open Data unter der Lizenz \texttt{dl-de/by-2-0} zum Download bereit
\parencite{unfallatlas_download}.

\subsection{Datenformat und Filterung}

Die Unfalldaten werden als gezippte CSV-Dateien im Koordinatenreferenzsystem
EPSG:25832 (\acrshort{utm} Zone 32N) bereitgestellt. Für das Projekt werden
die Daten nach dem Bundeslandschlüssel \texttt{ULAND=4} (Bremen) gefiltert.

\begin{table}[H]
      \centering
      \caption{Schweregrade der Unfälle im Unfallatlas}
      \label{tab:unfall_severity}
      \begin{tabular}{ccl}
            \toprule
            \textbf{UKATEGORIE} & \textbf{Schweregrad} & \textbf{Beschreibung}       \\
            \midrule
            1                   & \texttt{fatal}       & Unfall mit Getöteten        \\
            2                   & \texttt{severe}      & Unfall mit Schwerverletzten \\
            3                   & \texttt{minor}       & Unfall mit Leichtverletzten \\
            \bottomrule
      \end{tabular}
\end{table}

\subsection{Koordinatentransformation}

Da die Unfalldaten im projizierten System EPSG:25832 vorliegen, die Datenbank
jedoch EPSG:4326 (\acrshort{wgs84}) erwartet, erfolgt bei der Datenerfassung
eine automatische Reprojektion mithilfe der Python-Bibliothek GeoPandas
\parencite{geopandas}. Die Koordinatenspalten
(\path{XGCSWGS84}/\path{YGCSWGS84} bzw.\
\path{LINREFX}/\path{LINREFY}) werden automatisch erkannt.
Zudem wird das deutsche Dezimalkomma-Format
(z.\,B.\ \texttt{53,0793}) in Dezimalpunkt-Format konvertiert.

\subsection{Zeitraum und Umfang}

Es stehen Daten für die Jahre 2016\,--\,2024 zur Verfügung. Standardmäßig wird
das aktuellste verfügbare Jahr (2024) importiert. Die Anzahl der Unfälle in
Bremen variiert dabei je nach Jahr zwischen ca.\ 1.500 und 2.500
Datensätzen.
