% ============================================================
\chapter{Systemarchitektur}
\label{ch:architektur}
% ============================================================

Der Bremen Livability Index ist als klassische \textbf{Client-Server-Architektur}
mit drei Schichten konzipiert: einer räumlichen Datenbank, einem
\acrshort{rest}-Backend und einem plattformübergreifenden Frontend. Dieses
Kapitel gibt einen Überblick über den Gesamtaufbau und den Technologiestack.

\section{Architekturüberblick}
\label{sec:architektur_ueberblick}

Abbildung~\ref{fig:architektur_text} beschreibt die zentralen Komponenten und
deren Zusammenspiel. Durch die Trennung der Schichten lassen sich Backend und
Frontend unabhängig voneinander entwickeln, testen und deployen.

\begin{figure}[H]
  \centering
  \fbox{\parbox{0.9\textwidth}{%
      \textbf{Architekturdiagramm (textuell)}\\[0.5em]
      \texttt{%
        Flutter-App (Web / iOS / Android / Desktop)\\
        \quad $\downarrow$ HTTP/JSON\\
        FastAPI-Backend (Docker-Container auf Render.com)\\
        \quad $\downarrow$ SQL / PostGIS-Queries\\
        PostgreSQL 16 + PostGIS 3.4 (Neon.tech, Serverless)\\[0.5em]
        Nebenkomponenten:\\
        \quad Firebase Auth $\rightarrow$ Authentifizierung\\
        \quad Firebase Firestore $\rightarrow$ Favoritensynchronisation\\
        \quad Nominatim API $\rightarrow$ Adressgeokodierung\\
        \quad Overpass API $\rightarrow$ OSM-Datenerfassung (offline)
      }%
    }}
  \caption{Überblick der System\-architektur}
  \label{fig:architektur_text}
\end{figure}

\section{Technologiestack}
\label{sec:techstack}

\begin{table}[ht!]
  \centering
  \caption{Verwendete Technologien}
  \label{tab:techstack}
  \small
  \begin{tabularx}{\textwidth}{l>{\raggedright}p{4.5cm}>{\raggedright\arraybackslash}X}
    \toprule
    \textbf{Schicht} & \textbf{Technologie}                                                 & \textbf{Begründung} \\
    \midrule
    Datenbank
                     & PostgreSQL~16 + PostGIS~3.4
                     & Leistungsfähigste Open-Source-Lösung für räumliche Daten; native
    \texttt{GEOGRAPHY}-Unterstützung \parencite{postgis_docs}                                                     \\
    \addlinespace
    Backend
                     & FastAPI 0.115 (Python)
                     & Hohe Performance (ASGI), automatische OpenAPI-Doku,
    native Pydantic-Validierung \parencite{fastapi_docs}                                                          \\
    \addlinespace
    \acrshort{orm}
                     & SQLModel + GeoAlchemy2
                     & SQLAlchemy-Leistung mit Pydantic-Typisierung; PostGIS-Funktionen
    als Python-Objekte \parencite{sqlmodel_docs, geoalchemy2_docs}                                                \\
    \addlinespace
    Frontend
                     & Flutter 3.x (Dart)
                     & Plattformübergreifend aus einer Codebasis
    \parencite{flutter_docs}                                                                                      \\
    \addlinespace
    Karte
                     & flutter\_map 8.x
                     & Open-Source-Kartenwidget; CartoDB Voyager Tiles
    \parencite{flutter_map}                                                                                       \\
    \addlinespace
    State Mgmt.
                     & flutter\_bloc 9.x
                     & \acrshort{bloc}-Muster für reaktive, testbare Zustandsverwaltung
    \parencite{flutter_bloc}                                                                                      \\
    \addlinespace
    Auth
                     & Firebase Authentication
                     & Google, GitHub, Magic-Link, anonym; serverseitige Token-Validierung
    \parencite{firebase_docs}                                                                                     \\
    \addlinespace
    Hosting
                     & Render.com, Neon.tech
                     & Containerbasiertes Hosting + Serverless-DB im Free Tier; EU-Standort
    \parencite{render_docs, neontech}                                                                             \\
    \addlinespace
    Container
                     & Docker
                     & Reproduzierbare Build- und Deployment-Umgebung
    \parencite{docker_docs}                                                                                       \\
    \bottomrule
  \end{tabularx}
\end{table}

\section{Projektstruktur}

Das Projekt ist als öffentliches GitHub-Repository unter \\
\url{https://github.com/Milad9A/Bremen-Livability-Index} verfügbar und in
drei Hauptverzeichnisse gegliedert:

\begin{description}
  \item[\texttt{backend/}] Enthält das FastAPI-Backend mit den Unterordnern
        \texttt{app/} (Endpunkte, Modelle), \texttt{core/} (Datenbank, Scoring,
        Logging), \texttt{services/} (Geokodierung), \texttt{scripts/}
        (Datenerfassung) und \texttt{tests/} (Unit-Tests).
  \item[\texttt{frontend/bli/}] Enthält die Flutter-Anwendung mit der
        Verzeichnisstruktur \texttt{lib/core/} (Services, Theme, Widgets),
        \texttt{lib/features/} (Auth, Map, Favorites, Preferences, Onboarding)
        und plattformspezifischen Ordnern (\texttt{android/}, \texttt{ios/},
        \texttt{web/}, \texttt{macos/}, \texttt{windows/}, \texttt{linux/}).
  \item[\texttt{documentation/}] Enthält die vorliegende LaTeX-Dokumentation
        mit separaten Kapiteldateien und dem Literaturverzeichnis.
\end{description}

\section{Deployment-Architektur}
\label{sec:deployment}

Das Deployment erfolgt vollständig cloudbasiert über Infrastruktur im Free Tier
und wird über eine deklarative \texttt{render.yaml}-Konfiguration gesteuert:

\begin{itemize}
  \item \textbf{Backend}: Docker-Container auf Render.com (Web Service,
        Region Frankfurt). Der Container wird aus dem \texttt{backend/Dockerfile}
        gebaut und stellt den FastAPI-Server auf Port~8000 bereit. Die
        Umgebungsvariable \texttt{DATABASE\_URL} verbindet das Backend mit der
        Neon.tech-Datenbank.
  \item \textbf{Frontend}: Statische Flutter-Webanwendung auf Render.com
        (Static Site). Das Build-Skript \texttt{render\_build.sh} führt
        \texttt{flutter build web} aus. Eine SPA-Rewrite-Regel
        (\texttt{/* $\rightarrow$ /index.html}) stellt clientseitiges Routing
        sicher.
  \item \textbf{Datenbank}: PostgreSQL~16 mit PostGIS~3.4 auf Neon.tech
        (Serverless). Die Datenbank skaliert automatisch und bietet
        branching-fähige Entwicklungsumgebungen.
  \item \textbf{E-Mail-Redirect}: Firebase Hosting leitet Magic-Link-URLs
        an die Flutter-App weiter.
\end{itemize}
