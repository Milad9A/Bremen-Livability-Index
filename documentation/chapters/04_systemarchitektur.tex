% !TEX root = ../main.tex
% ============================================================
\chapter{Systemarchitektur}
\label{ch:architektur}
% ============================================================

Der Bremen Livability Index ist als klassische \textbf{Client-Server-Architektur}
mit drei Schichten konzipiert: einer räumlichen Datenbank, einem
\acrshort{rest}-Backend und einem plattformübergreifenden Frontend. Dieses
Kapitel gibt einen Überblick über den Gesamtaufbau und den Technologiestack.

\section{Architekturüberblick}
\label{sec:architektur_ueberblick}

Abbildung~\ref{fig:architektur} zeigt die zentralen Komponenten und
deren Zusammenspiel. Durch die Trennung der Schichten lassen sich Backend und
Frontend unabhängig voneinander entwickeln, testen und deployen.

Die Kernfunktionalität -- räumliche Näheanalysen mittels PostGIS-Funktionen wie
\texttt{ST\_DWithin} und \texttt{GEOGRAPHY}-Datentypen -- erfordert eine
vollwertige räumliche Datenbank, die Firebase/Firestore nicht bieten kann.
Gleichzeitig stellt Firebase Authentication bewährte OAuth- und
Magic-Link-Flows bereit, deren Eigenimplementierung unverhältnismäßig aufwändig
wäre. Die Architektur trennt daher bewusst: \emph{geodatenintensive Logik}
verbleibt im PostGIS-Backend, während \emph{Identitätsverwaltung und
  Favoritensynchronisation} über Firebase-Dienste abgewickelt werden.

\begin{figure}[H]
  \centering
  \begin{tikzpicture}[
    node distance=1.6cm and 1.8cm,
    every node/.style={font=\small},
    box/.style={draw, rounded corners=3pt, minimum width=4.8cm,
                minimum height=0.9cm, align=center, fill=#1!12,
                draw=#1!60, text=#1!80!black},
    box/.default=teal,
    svc/.style={draw, rounded corners=3pt, minimum width=3.4cm,
                minimum height=0.75cm, align=center, fill=#1!10,
                draw=#1!50, text=#1!70!black, font=\small},
    svc/.default=gray,
    arr/.style={-{Stealth[length=5pt]}, thick, color=#1!70},
    arr/.default=teal,
    lbl/.style={font=\scriptsize\sffamily, text=black!60},
  ]
    % ── Main stack ──
    \node[box=teal]  (fe)  {Flutter-App\\[-2pt]
                             \scriptsize Web\,/\,iOS\,/\,Android\,/\,Desktop};
    \node[box=blue,  below=of fe]  (be)  {FastAPI-Backend\\[-2pt]
                             \scriptsize Docker auf Render.com};
    \node[box=violet, below=of be] (db)  {PostgreSQL\,16 + PostGIS\,3.4\\[-2pt]
                             \scriptsize Neon.tech (Serverless)};

    % ── Arrows main ──
    \draw[arr=teal]   (fe) -- node[lbl, right, xshift=2pt] {HTTP\,/\,JSON} (be);
    \draw[arr=blue]   (be) -- node[lbl, right, xshift=2pt] {SQL\,/\,PostGIS} (db);

    % ── Side services (Firebase group, aligned to Flutter row) ──
    \node[svc=orange, right=2.2cm of fe, yshift=0.4cm] (auth) {Firebase Auth\\[-2pt]
                             \scriptsize Authentifizierung};
    \node[svc=orange, below=0.45cm of auth] (fire) {Firestore\\[-2pt]
                             \scriptsize Favoriten\,\&\,Präferenzen};
    % ── Side services (External APIs, aligned to Backend row) ──
    \node[svc=gray,   right=2.2cm of be, yshift=0.4cm] (nom)  {Nominatim API\\[-2pt]
                             \scriptsize Geokodierung};
    \node[svc=gray,   below=0.45cm of nom] (over) {Overpass API\\[-2pt]
                             \scriptsize OSM-Daten (offline)};

    % ── Arrows side ──
    \draw[arr=orange] (fe.east) -- (auth.west);
    \draw[arr=orange] (fe.east) ++(0,-0.15) -| ([xshift=-4pt]fire.west) -- (fire.west);
    \draw[arr=gray]   (be.east) -- (nom.west);
    \draw[arr=gray]   (be.east) ++(0,-0.15) -| ([xshift=-4pt]over.west) -- (over.west);

    % ── Background groups ──
    \begin{scope}[on background layer]
      \node[fit=(auth)(fire), fill=orange!5, draw=orange!25,
            rounded corners=5pt, inner sep=6pt] {};
      \node[fit=(nom)(over), fill=gray!5, draw=gray!25,
            rounded corners=5pt, inner sep=6pt] {};
    \end{scope}
  \end{tikzpicture}
  \caption{Systemarchitektur des Bremen Livability Index}
  \label{fig:architektur}
\end{figure}

\section{Projektstruktur}

Das Projekt ist als öffentliches GitHub-Repository unter \\
\url{https://github.com/Milad9A/Bremen-Livability-Index} verfügbar und in
drei Hauptverzeichnisse gegliedert:

\begin{description}
  \item[\texttt{backend/}] Enthält das FastAPI-Backend mit den Unterordnern
        \texttt{app/} (Endpunkte, Modelle), \texttt{core/} (Datenbank, Scoring,
        Logging), \texttt{services/} (Geokodierung), \texttt{scripts/}
        (Datenerfassung) und \texttt{tests/} (Unit-Tests).
  \item[\texttt{frontend/bli/}] Enthält die Flutter-Anwendung mit der
        Verzeichnisstruktur \texttt{lib/core/} (Services, Theme, Widgets),
        \texttt{lib/features/} (Auth, Map, Favorites, Preferences, Onboarding)
        und plattformspezifischen Ordnern (\texttt{android/}, \texttt{ios/},
        \texttt{web/}, \texttt{macos/}, \texttt{windows/}, \texttt{linux/}).
  \item[\texttt{documentation/}] Enthält die vorliegende LaTeX-Dokumentation
        mit separaten Kapiteldateien und dem Literaturverzeichnis.
\end{description}

\section{Technologiestack}
\label{sec:techstack}

Tabelle~\ref{tab:techstack} gibt einen Überblick über alle eingesetzten
Technologien und begründet deren Auswahl.

\begin{table}[ht!]
  \centering
  \caption{Verwendete Technologien}
  \label{tab:techstack}
  \small
  \begin{tabularx}{\textwidth}{l>{\raggedright}p{4.5cm}>{\raggedright\arraybackslash}X}
    \toprule
    \textbf{Schicht} & \textbf{Technologie}                                                 & \textbf{Begründung} \\
    \midrule
    Datenbank
                     & PostgreSQL~16 + PostGIS~3.4
                     & Leistungsfähigste Open-Source-Lösung für räumliche Daten; native
    \texttt{GEOGRAPHY}-Unterstützung \parencite{postgis_docs}                                                     \\
    \addlinespace
    Backend
                     & FastAPI 0.115 (Python)
                     & Hohe Performance (ASGI), automatische OpenAPI-Doku,
    native Pydantic-Validierung \parencite{fastapi_docs}                                                          \\
    \addlinespace
    Validierung
                     & Pydantic 2.x
                     & Laufzeit-Datenvalidierung über Python-Typannotationen; bildet die
    Grundlage für FastAPI-Request-Validierung und SQLModel-Typisierung                                            \\
    \addlinespace
    \acrshort{orm}
                     & SQLModel + GeoAlchemy2
                     & SQLAlchemy-Leistung mit Pydantic-Typisierung; PostGIS-Funktionen
    als Python-Objekte \parencite{sqlmodel_docs, geoalchemy2_docs}                                                \\
    \addlinespace
    Frontend
                     & Flutter 3.x (Dart)
                     & Plattformübergreifend aus einer Codebasis
    \parencite{flutter_docs}                                                                                      \\
    \addlinespace
    Karte
                     & flutter\_map 8.x
                     & Open-Source-Kartenwidget; CartoDB Voyager Tiles
    \parencite{flutter_map}                                                                                       \\
    \addlinespace
    State Mgmt.
                     & flutter\_bloc 9.x
                     & \acrshort{bloc}-Muster für reaktive, testbare Zustandsverwaltung
    \parencite{flutter_bloc}                                                                                      \\
    \addlinespace
    Auth
                     & Firebase Authentication
                     & Google, GitHub, Magic-Link, anonym; serverseitige Token-Validierung
    \parencite{firebase_docs}                                                                                     \\
    \addlinespace
    Hosting
                     & Render.com, Neon.tech
                     & Containerbasiertes Hosting + Serverless-DB im Free Tier; EU-Standort
    \parencite{render_docs, neontech}                                                                             \\
    \addlinespace
    Container
                     & Docker
                     & Reproduzierbare Build- und Deployment-Umgebung
    \parencite{docker_docs}                                                                                       \\
    \bottomrule
  \end{tabularx}
\end{table}

\section{Deployment-Architektur}
\label{sec:deployment}

Das Deployment erfolgt vollständig cloudbasiert über Infrastruktur im Free Tier
und wird über eine deklarative \texttt{render.yaml}-Konfiguration gesteuert:

\begin{itemize}
  \item \textbf{Backend}: Docker-Container auf Render.com (Web Service,
        Region Frankfurt). Der Container wird aus dem \texttt{backend/Dockerfile}
        gebaut und stellt den FastAPI-Server auf Port~8000 bereit. Die
        Umgebungsvariable \texttt{DATABASE\_URL} verbindet das Backend mit der
        Neon.tech-Datenbank.
  \item \textbf{Frontend (Web)}: Statische Flutter-Webanwendung auf Render.com
        (Static Site). Das Build-Skript \texttt{render\_build.sh} führt
        \texttt{flutter build web} aus. Eine SPA-Rewrite-Regel
        (\texttt{/* $\rightarrow$ /index.html}) stellt clientseitiges Routing
        sicher.
  \item \textbf{Frontend (Mobile \& Desktop)}: Ein GitHub-Actions-Workflow
        (\texttt{build-release.yml}) wird nach jedem erfolgreichen Frontend-Test
        auf dem \texttt{master}-Branch automatisch ausgelöst. Er baut die App
        für Android (APK), Windows (ZIP), Linux (tarball) und macOS (ZIP) und
        veröffentlicht alle vier Artefakte als einheitliches GitHub~Release mit
        Zeitstempel-Tag. Desktop-Builds sind ad-hoc-signiert und erfordern ggf.
        eine Sicherheitsausnahme beim ersten Start.
  \item \textbf{Datenbank}: PostgreSQL~16 mit PostGIS~3.4 auf Neon.tech
        (Serverless). Die Datenbank skaliert automatisch und bietet
        branching-fähige Entwicklungsumgebungen.
  \item \textbf{E-Mail-Redirect}: Firebase Hosting leitet Magic-Link-URLs
        an die Flutter-App weiter.
\end{itemize}
