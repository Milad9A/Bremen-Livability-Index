% ============================================================
\chapter{Datenerfassung}
\label{ch:datenerfassung}
% ============================================================

Die Befüllung der Datenbank erfolgt automatisiert über Python-Skripte, die
im Verzeichnis \path{backend/scripts/data_ingestion/} organisiert sind.
Der Einstiegspunkt \texttt{ingest\_all\_data.py} ruft beide
Ingestion-Module nacheinander auf.

\section{OSM-Datenerfassung}
\label{sec:osm_ingestion}

Die Erfassung der 20 \acrshort{osm}-Kategorien erfolgt über die
Python-Bibliothek \texttt{overpy} \parencite{overpy}, einen Wrapper für die
Overpass API \parencite{overpass_api}.

\subsection{Ablauf}

Für jede Kategorie wird folgende Pipeline durchlaufen:

\begin{enumerate}
      \item \textbf{Overpass-Abfrage}: Eine Overpass-QL-Abfrage wird mit dem
            Bounding-Box-Filter für Bremen formuliert und an die Overpass API
            gesendet.
      \item \textbf{Retry-Logik}: Bei Überlastung der API (HTTP~429 oder
            Gateway-Timeout) wird ein exponentieller Backoff mit Jitter
            angewendet:
            \[
                  \text{Wartezeit} = 10 \cdot 2^{\text{Versuch}} + r
                  \qquad r \sim \mathcal{U}(0, 5)
            \]
            Maximal werden 5 Versuche unternommen.
      \item \textbf{Tabelle leeren}: \texttt{TRUNCATE TABLE gis\_data.xxx CASCADE}
            entfernt alle bestehenden Daten, um eine idempotente
            Neuerfassung zu gewährleisten.
      \item \textbf{Bulk-Insert}: Die empfangenen Nodes, Ways oder Relationen
            werden einzeln in die zugehörige Tabelle eingefügt.
\end{enumerate}

\subsection{Geometriekonvertierung}

Die von der Overpass API zurückgegebenen Objekte werden je nach Typ
unterschiedlich verarbeitet:

\begin{description}
      \item[Punkte (Nodes)] werden mit \texttt{ST\_MakePoint(lon, lat)} in einen
            PostGIS-Point konvertiert und mit \texttt{ST\_SetSRID(..., 4326)} dem
            \acrshort{wgs84}-System zugewiesen.
      \item[Polygone (Ways)] Die Koordinaten der Way-Nodes werden zu einem
            WKT-String \texttt{POLYGON(({coords}))} zusammengesetzt. Falls der
            erste und letzte Node nicht identisch sind, wird der Ring automatisch
            geschlossen.
      \item[Linienzüge (Ways)] Analog werden die Koordinaten als
            \texttt{LINESTRING({coords})} zusammengesetzt.
\end{description}

Alle Geometrien werden abschließend nach \texttt{GEOGRAPHY(type, 4326)} gecastet.
Die Datenbankverbindung wird über \texttt{psycopg2} direkt aufgebaut
(ohne ORM), da Bulk-Inserts so effizienter sind.

% ──────────────────────────────────────────────────────────────

\section{Unfallatlas-Datenerfassung}
\label{sec:unfallatlas_ingestion}

Die Erfassung der Unfalldaten \parencite{unfallatlas_download} umfasst
mehrere Schritte, da die Quelldaten in einem anderen Format und
Koordinatensystem vorliegen.

\subsection{Download und Extraktion}

Die Daten werden als gezippte CSV-Datei von der Open-Data-Plattform des
Landes Nordrhein-Westfalen heruntergeladen:

\begin{lstlisting}[basicstyle=\ttfamily\small,breaklines=true,frame=none]
https://www.opengeodata.nrw.de/produkte/transport_verkehr/
  unfallatlas/Unfallorte2024_EPSG25832_CSV.zip
\end{lstlisting}

\subsection{Filterung nach Bremen}

Die CSV-Datei enthält Unfalldaten für ganz Deutschland. Die Filterung auf
Bremen erfolgt anhand des Bundeslandschlüssels:

\begin{lstlisting}[style=python,caption={Filterung auf Bremen},label={lst:filter_bremen}]
# ULAND = 4 entspricht dem Bundesland Bremen
df = df[df["ULAND"] == 4]
\end{lstlisting}

\subsection{Koordinatentransformation}

Die Unfalldaten liegen im projizierten Koordinatensystem EPSG:25832
(\acrshort{utm} Zone 32N) vor \parencite{epsg25832}. Für die Speicherung in
der PostGIS-Datenbank (EPSG:4326) ist eine Reprojektion erforderlich.

Das Skript erkennt automatisch die Koordinatenspalten
(\path{XGCSWGS84}/\path{YGCSWGS84}
bzw.\ \path{LINREFX}/\path{LINREFY}),
konvertiert das deutsche Dezimalkomma-Format und reprojiziert die Daten
mithilfe von GeoPandas \parencite{geopandas} nach EPSG:4326.

\subsection{Schweregrad-Mapping}

Das Feld \texttt{UKATEGORIE} wird in lesbare Schweregrade konvertiert
(siehe Tabelle~\ref{tab:unfall_severity} in Kapitel~\ref{ch:datenquellen}):
$1 \rightarrow \texttt{fatal}$, $2 \rightarrow \texttt{severe}$,
$3 \rightarrow \texttt{minor}$.

\subsection{Verfügbare Jahrgänge}

Es stehen Daten für die Jahrgänge 2016\,--\,2024 zur Verfügung. Das
Ingestion-Skript akzeptiert das gewünschte Jahr als Parameter und versucht,
bei Nicht-Verfügbarkeit automatisch das nächstältere Jahr zu verwenden.
