% !TEX root = ../main.tex
% ============================================================
\chapter{Fazit und Ausblick}
\label{ch:fazit}
% ============================================================

\section{Zusammenfassung}

Im Rahmen des Moduls \textit{Geodatenverarbeitung} an der Hochschule Bremen
wurde mit dem \textbf{Bremen Livability Index} ein vollständiges
Geoinformationssystem konzipiert, implementiert und als produktionstaugliche
Anwendung bereitgestellt. Ausgangspunkt war der Wunsch, heterogene, frei
verfügbare Geodaten -- aus OpenStreetMap und dem Unfallatlas -- zu einem
einheitlichen, feinräumigen Lebensqualitätsindex zu verdichten. Über eine
vollständig automatisierte Ingestion-Pipeline wurden mehr als 60.000
Objekte aus 20 thematischen Kategorien in eine PostGIS-Datenbank geladen
und mit GiST-Indizes für Echtzeit-Radiusabfragen optimiert. Das FastAPI-Backend
berechnet Scores innerhalb von Millisekunden und liefert die umliegenden
\acrshort{gis}-Features als GeoJSON zurück. Die Flutter-Applikation stellt
diese Ergebnisse plattformübergreifend -- Web, iOS, Android, Desktop --
auf einer interaktiven Karte dar und erlaubt individuelle Präferenzgewichtung.
Authentifizierung und Favoritensynchronisation werden über Firebase abgewickelt,
während die domänenspezifische Geo-Logik ausschließlich im Backend verbleibt.
Das Projekt zeigt, dass sich aus frei verfügbaren Geodaten mit vertretbarem
Aufwand ein praxistaugliches, feinräumiges Analysewerkzeug realisieren lässt.

\section{Beantwortung der Zielsetzung}

Die zentrale Fragestellung -- ob sich aus frei verfügbaren Geodaten ein
aussagekräftiger, feinräumiger Lebensqualitätsindex für Bremen ableiten
lässt -- kann positiv beantwortet werden. Alle fünf in
Kapitel~\ref{ch:einleitung} formulierten Projektziele wurden erreicht:
\emph{(1)}~Über 60.000 Geodatenobjekte werden automatisiert erfasst und
aktualisiert; \emph{(2)}~der Livability Score (0\,--\,100) wird für
beliebige Bremer Koordinaten in Echtzeit berechnet; \emph{(3)}~die
Faktoraufschlüsselung macht positive und negative Einflüsse transparent
nachvollziehbar; \emph{(4)}~die Kartenanwendung visualisiert die umgebenden
\acrshort{gis}-Features interaktiv; und \emph{(5)}~Nutzer können die
Gewichtung der Faktoren ihren persönlichen Präferenzen anpassen. Die
exemplarischen Standortbewertungen (Kapitel~\ref{ch:ergebnisse}) bestätigen,
dass das Modell plausible Ergebnisse liefert: urban dichte, gut versorgte
Standorte erzielen deutlich höhere Scores als periphere Industrie- oder
Verkehrsflächen.

\section{Ausblick}

Konkrete technische Erweiterungsmöglichkeiten -- von der Skalierung auf
weitere Städte über Echtzeitdaten bis hin zu datengetriebener Kalibrierung --
wurden bereits in Abschnitt~\ref{sec:verbesserungen} identifiziert. Darüber
hinaus ergeben sich weiterreichende Perspektiven.

Der gewählte Architekturansatz -- automatisierte OSM-Ingestion, konfigurierbares
Scoring-Schema und plattformübergreifendes Frontend -- ist nicht auf das Thema
Lebensqualität beschränkt. Dieselbe Pipeline ließe sich mit angepassten
Kategorien und Gewichtungen auf verwandte Fragestellungen übertragen, etwa
die Bewertung von Gewerbestandorten, barrierefreier Infrastruktur oder
Schulwegqualität.

Für eine institutionelle Nutzung in der Stadtplanung oder Immobilienwirtschaft
wäre eine empirische Validierung der Gewichtungen unerlässlich -- beispielsweise
durch Korrelation der berechneten Scores mit Mietpreisspiegeln,
Zufriedenheitsbefragungen oder soziodemographischen Indikatoren. Die
Bereitstellung einer öffentlichen \acrshort{api} könnte zudem die Integration
in bestehende Geoinformations- und Planungswerkzeuge erleichtern und den
Anwendungskreis über das akademische Umfeld hinaus erweitern.
