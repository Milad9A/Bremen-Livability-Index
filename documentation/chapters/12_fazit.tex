% !TEX root = ../main.tex
% ============================================================
\chapter{Fazit und Ausblick}
\label{ch:fazit}
% ============================================================

\section{Zusammenfassung}

Im Rahmen des Moduls \textit{Geodatenverarbeitung} an der Hochschule Bremen
wurde mit dem \textbf{Bremen Livability Index} ein vollständiges
Geoinformationssystem konzipiert, implementiert und als produktionstaugliche
Webanwendung bereitgestellt. Das System erfüllt alle fünf in
Kapitel~\ref{ch:einleitung} definierten Ziele: automatisierte Erfassung
von über 60.000 Geodaten-Objekten, Echtzeit-Scoring (0\,--\,100) für
beliebige Standorte, transparente Faktoraufschlüsselung mit
GeoJSON-Visualisierung, interaktive Kartenanwendung und individuelle
Gewichtung durch den Nutzer.

\section{Beantwortung der Zielsetzung}

Die zentrale Fragestellung -- ob sich aus frei verfügbaren Geodaten ein
aussagekräftiger, feinräumiger Lebensqualitätsindex für Bremen ableiten
lässt -- kann positiv beantwortet werden. Die exemplarischen Ergebnisse
(Kapitel~\ref{ch:ergebnisse}) zeigen plausible Unterschiede zwischen
verschiedenen Standorttypen.

\section{Ausblick}

Vielversprechende Weiterentwicklungen umfassen die Skalierung auf weitere
Städte, die Integration von Echtzeitdaten (Luftqualität, Lärmkarten) und
die empirische Validierung der Gewichtungen durch Nutzerbefragungen
oder maschinelles Lernen.
