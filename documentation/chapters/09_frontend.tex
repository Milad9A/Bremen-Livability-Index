% ============================================================
\chapter{Frontend}
\label{ch:frontend}
% ============================================================

Das Frontend ist als plattformübergreifende Anwendung mit dem
UI-Framework Flutter \parencite{flutter_docs} implementiert und unterstützt
Web, iOS, Android, macOS, Windows und Linux.

\section{Architektur und State Management}
\label{sec:frontend_architektur}

Die Anwendung folgt dem \acrfull{bloc}-Entwurfsmuster
(vgl.\ Abschnitt~\ref{sec:bloc_pattern}), das die Geschäftslogik
vollständig von der Darstellungsschicht trennt. Die Architektur gliedert
sich in die folgenden Module:

\begin{description}
      \item[\texttt{lib/core/}] Querschnittskomponenten:
            \path{ApiService} (HTTP-Client via Dio),
            \path{DeepLinkService}, Theme-Definitionen
            (\path{AppTheme}, \path{AppColors},
            \path{AppTextStyles}) und wiederverwendbare Widgets.
      \item[\texttt{lib/features/auth/}] Authentifizierungslogik:
            \texttt{AuthBloc}, \texttt{AuthService}, Login-Screens.
      \item[\texttt{lib/features/map/}] Kern-Feature:
            \texttt{MapBloc} (Karteninteraktionen), \texttt{MapScreen},
            \texttt{ScoreCardView}, \texttt{FloatingSearchBar}.
      \item[\texttt{lib/features/favorites/}] Favoritenverwaltung:
            \texttt{FavoritesBloc}, Datenmodelle.
      \item[\texttt{lib/features/preferences/}] Nutzerpräferenzen:
            \texttt{PreferencesBloc}, \texttt{PreferencesScreen},
            \texttt{PreferencesService}.
      \item[\texttt{lib/features/onboarding/}] Startbildschirm.
\end{description}

Alle Events und States werden mithilfe des Code-Generators \texttt{Freezed}
als \textit{Sealed Unions} definiert. Dies erzwingt typsichere, vollständige
Pattern-Matching-Behandlung in der UI und verhindert undefinierte Zustände.

\section{Kartenansicht}
\label{sec:kartenansicht}

Die Kartenansicht bildet die zentrale Benutzeroberfläche. Sie basiert auf dem
Widget \texttt{FlutterMap} \parencite{flutter_map} mit \textbf{CartoDB Voyager}
Tiles -- einem schlichten Kartenstil ohne störende POI-Beschriftungen.

\begin{itemize}
      \item \textbf{Startzentrum}: $53{,}0793^{\circ}\,\text{N}$, $8{,}8017^{\circ}\,\text{E}$
            (Bremen Innenstadt)
      \item \textbf{Startzoom}: 13
      \item \textbf{Interaktion}: Ein Tipp auf die Karte löst ein
            \texttt{MapTapped}-Event aus, das den \texttt{MapBloc} veranlasst,
            den \texttt{ApiService} anzufragen. Der berechnete Score und die
            nahen Features werden anschließend als Marker und Score-Karte
            dargestellt.
\end{itemize}

Jede Feature-Kategorie erhält einen eigenen Marker mit einer
standardisierten Farbpalette aus \texttt{AppColors}, sodass der Nutzer
die Art der nahen Einrichtungen visuell unterscheiden kann.

\section{Benutzeroberfläche -- Liquid Glass Design}
\label{sec:ui_design}

Das UI-Design folgt einem \textit{Liquid Glass}-Konzept: Die Karte nimmt
den gesamten Bildschirm ein, während alle interaktiven Elemente
(Suchleiste, Score-Karte, Präferenzen) als halbtransparente, gefrostete
Glasflächen über der Karte schweben. Optische Effekte wie leichte
Vergrößerung, Verzerrung und haptisches Feedback auf Mobilgeräten
runden das Interaktionserlebnis ab.

\section{Authentifizierung}
\label{sec:auth}

Die Authentifizierung wird über Firebase Authentication \parencite{firebase_docs}
abgewickelt und unterstützt mehrere Anmeldemethoden:

\begin{table}[H]
      \centering
      \caption{Unterstützte Anmeldemethoden nach Plattform}
      \label{tab:auth_methods}
      \begin{tabularx}{\textwidth}{lX}
            \toprule
            \textbf{Plattform} & \textbf{Anmeldemethoden}                       \\
            \midrule
            Web, iOS, Android
                               & Google, GitHub, E-Mail-Magic-Link, Anonym      \\
            macOS, Windows, Linux
                               & Nur als Gast (Firebase Auth wird übersprungen) \\
            \bottomrule
      \end{tabularx}
\end{table}

Auf Desktop-Plattformen wird die Firebase-Authentifizierung vollständig
umgangen, da macOS-Keychains durch Sandboxing eingeschränkt sind und
Windows/Linux keine nativen OAuth-Desktop-Flows unterstützen. Stattdessen
wird ein lokales \texttt{AppUser.guest()}-Objekt ohne Firebase-Interaktion
erzeugt.

\subsection{Cross-Device E-Mail-Link-Flow}

Wenn ein Nutzer einen Magic Link auf einem anderen Gerät öffnet als dem,
auf dem er die Anmeldung initiiert hat, ist die E-Mail-Adresse nicht
im lokalen Speicher hinterlegt. In diesem Fall erkennt der
\texttt{DeepLinkService} den \texttt{oobCode}-Parameter in der URL,
und der \texttt{AuthBloc} navigiert den Nutzer zu einem
\texttt{EmailLinkPromptScreen}, auf dem er seine E-Mail-Adresse erneut
eingeben kann.

\section{Favoritenverwaltung}
\label{sec:favorites}

Nutzer können analysierte Standorte als Favoriten speichern. Die
Datenhaltung ist zweistufig:

\begin{itemize}
      \item \textbf{Angemeldete Nutzer}: Synchronisation über Firebase Firestore
            und das Backend (\path{/users/{user_id}/favorites}).
      \item \textbf{Gäste}: Lokale Speicherung über \texttt{SharedPreferences}
            (Key-Value-Store des Geräts).
\end{itemize}

Der \texttt{FavoritesBloc} abstrahiert die Speicherstrategie und stellt
eine einheitliche Schnittstelle für das UI bereit.

\section{Adresssuche}

Die \texttt{FloatingSearchBar} ermöglicht die Suche nach Adressen.
Eingaben werden über den \texttt{ApiService} an den
\texttt{/geocode}-Endpunkt des Backends weitergeleitet, der wiederum die
Nominatim-\acrshort{api} abfragt. Die Suchergebnisse werden als
Drop-down-Liste angezeigt; bei Auswahl wird die Karte zur
entsprechenden Koordinate navigiert und automatisch eine
Score-Berechnung ausgelöst.
