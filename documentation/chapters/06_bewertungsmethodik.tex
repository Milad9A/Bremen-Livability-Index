% !TEX root = ../main.tex
% ============================================================
\chapter{Bewertungsmethodik}
\label{ch:bewertung}
% ============================================================

Das Kernstück des Bremen Livability Index ist der Bewertungsalgorithmus, der
für einen gegebenen geographischen Punkt einen \textbf{Livability Score}
zwischen 0 und 100 berechnet.

\section{Bewertungsformel}
\label{sec:formel}

Der Score setzt sich aus einem Basiswert, der Summe positiver Faktoren und
der Summe negativer Faktoren zusammen:

\begin{equation}
  \label{eq:score}
  \text{Score} = \text{clamp}\!\left(
  \underbrace{S_{\text{base}}}_{= 40}
  + \sum_{i=1}^{9} w_i \cdot f_i^{+}
  - \sum_{j=1}^{11} w_j \cdot f_j^{-}
  ,\; 0,\; 100
  \right)
\end{equation}

Dabei ist $S_{\text{base}} = 40$ ein neutraler Ausgangswert,
$f_i^{+}$ bzw.\ $f_j^{-}$ die Einzelscores der positiven/negativen Faktoren
und $w \in \{0{,}0;\; 0{,}5;\; 1{,}0;\; 1{,}5\}$ der nutzerspezifische
Gewichtungsmultiplikator (\textit{ImportanceLevel}: \texttt{excluded},
\texttt{low}, \texttt{medium}, \texttt{high}).

Faktoren mit hohen Zählergebnissen verwenden logarithmische Skalierung
$f(n) = \min(f_{\max},\; \ln(1{+}n) \cdot k)$, die übrigen lineare
Skalierung $f(n) = \min(f_{\max},\; n \cdot k)$.

\section{Positive Faktoren}
\label{sec:positive_faktoren}

\begin{table}[H]
  \centering
  \caption{Positive Einflussfaktoren (Summe Max.: 60)}
  \label{tab:positive}
  \scriptsize
  \begin{tabularx}{\textwidth}{p{2.8cm}ccX}
    \toprule
    \textbf{Faktor} & \textbf{Max.} & \textbf{Radius} & \textbf{Formel}                                                \\
    \midrule
    Grünflächen     & 14            & 175\,m          & $\min(9, \ln(1{+}n_B) \cdot 2{,}0) + \min(5, n_P \cdot 2{,}5)$ \\
    Nahversorgung   & 10            & 550\,m          & $\min(10, \ln(1{+}n) \cdot 2{,}8)$                             \\
    ÖPNV            & 8             & 450\,m          & $\min(8, \ln(1{+}n) \cdot 3{,}5)$                              \\
    Gesundheit      & 6             & 700\,m          & $\min(6, n \cdot 2{,}5)$                                       \\
    Fahrrad         & 6             & 275\,m          & $\min(6, \ln(1{+}n) \cdot 2{,}5)$                              \\
    Bildung         & 5             & 500\,m          & $\min(5, n \cdot 1{,}5)$                                       \\
    Sport/Freizeit  & 4             & 700\,m          & $\min(4, \ln(1{+}n) \cdot 1{,}8)$                              \\
    Kultur          & 4             & 500\,m          & $\min(4, n \cdot 2{,}0)$                                       \\
    Fußgänger       & 3             & 275\,m          & $\min(3, \ln(1{+}n) \cdot 1{,}2)$                              \\
    \bottomrule
  \end{tabularx}
\end{table}

\section{Negative Faktoren}
\label{sec:negative_faktoren}

\begin{table}[H]
  \centering
  \caption{Negative Einflussfaktoren (Summe Max.: 57)}
  \label{tab:negative}
  \scriptsize
  \begin{tabularx}{\textwidth}{p{2.8cm}ccX}
    \toprule
    \textbf{Faktor} & \textbf{Strafe} & \textbf{Radius} & \textbf{Typ}             \\
    \midrule
    Industriegebiet & 10              & 150\,m          & Binär                    \\
    Unfälle         & 8               & 120\,m          & $\min(8, n \cdot 2{,}0)$ \\
    Flughafen       & 7               & 600\,m          & Binär                    \\
    Hauptstraßen    & 6               & 60\,m           & Binär                    \\
    Lärmquellen     & 6               & 75\,m           & $\min(6, n \cdot 2{,}0)$ \\
    Abfall          & 5               & 250\,m          & Binär                    \\
    Eisenbahn       & 5               & 100\,m          & Binär                    \\
    Tankstelle      & 3               & 75\,m           & Binär                    \\
    Strom           & 3               & 75\,m           & Binär                    \\
    Baustelle       & 2               & 125\,m          & Binär                    \\
    Großparkplatz   & 2               & 50\,m           & Binär                    \\
    \bottomrule
  \end{tabularx}
\end{table}

Binäre Faktoren vergeben die volle Strafe, sobald mindestens ein Objekt
im Radius vorhanden ist. Zählerbasierte (Unfälle, Lärm) steigen
proportional, sind aber nach oben begrenzt. Die Suchradien
(50\,--\,700\,m) spiegeln den Einflussbereich der Faktoren wider;
alle Abfragen nutzen \texttt{ST\_DWithin} auf dem \acrshort{wgs84}-Ellipsoid.

\section{Score-Interpretation und Implementierung}

Die App visualisiert den Score durch eine dreistufige Farbcodierung:
$\geq 70$ Grün (gut), $50$--$69$ Orange (mittel), $< 50$ Rot (schlecht).
Der numerische Wert wird direkt angezeigt.

Der Algorithmus ist in \texttt{LivabilityScorer} (\texttt{core/scoring.py})
implementiert. Jeder Faktor wird durch eine eigene statische Methode berechnet;
\texttt{calculate\_score} aggregiert alle Beiträge zu einem
\texttt{LivabilityScoreResponse}-Objekt.
