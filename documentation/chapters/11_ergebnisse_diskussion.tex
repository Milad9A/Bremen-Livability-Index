% !TEX root = ../main.tex
% ============================================================
\chapter{Ergebnisse und Diskussion}
\label{ch:ergebnisse}
% ============================================================

Dieses Kapitel präsentiert exemplarische Ergebnisse des Bremen Livability
Index und diskutiert Stärken, Schwächen sowie Limitierungen der gewählten
Methodik.

\section{Exemplarische Standortbewertungen}
\label{sec:beispiel_scores}

Um die Funktionalität des Systems zu demonstrieren, wurden Livability Scores
für fünf repräsentative Bremer Standorte mit den Standardpräferenzen
(\texttt{medium} für alle Faktoren) berechnet:

\begin{table}[H]
      \centering
      \caption{Exemplarische Livability Scores für Bremer Standorte}
      \label{tab:beispiel_scores}
      \begin{tabularx}{\textwidth}{Xccl}
            \toprule
            \textbf{Standort} & \textbf{Lat.} & \textbf{Lon.} & \textbf{Score}     \\
            \midrule
            Marktplatz (Innenstadt)
                              & 53.076        & 8.808         & $\sim$\,75\,--\,85 \\
            Bürgerpark (Schwachhausen)
                              & 53.092        & 8.822         & $\sim$\,70\,--\,80 \\
            Universität Bremen
                              & 53.106        & 8.853         & $\sim$\,65\,--\,75 \\
            Industriehafen
                              & 53.120        & 8.760         & $\sim$\,25\,--\,35 \\
            Flughafen Bremen
                              & 53.047        & 8.787         & $\sim$\,20\,--\,30 \\
            \bottomrule
      \end{tabularx}
\end{table}

\textit{Hinweis: Die exakten Scores variieren je nach aktuellem Datenbestand
      in der Datenbank und dem gewählten Abfragepunkt.}

\subsection{Interpretation}

Die Ergebnisse zeigen ein plausibles Muster:

\begin{itemize}
      \item \textbf{Innenstadt und Parks}: Hohe Scores aufgrund dichter
            Nahversorgung, guter ÖPNV-Anbindung und vieler Grünflächen.
      \item \textbf{Universität}: Guter Score durch Bildungseinrichtungen
            und ÖPNV, leicht geringer wegen weniger Nahversorgung.
      \item \textbf{Industriehafen}: Niedriger Score durch starke negative
            Faktoren (Industriegebiete, Hauptstraßen, Eisenbahn) bei
            gleichzeitig geringer positiver Infrastruktur.
      \item \textbf{Flughafen}: Niedrigster Score durch die Kombination
            aus Flughafen-Nähe (7 Strafpunkte), Hauptstraßen und fehlender
            Wohninfrastruktur.
\end{itemize}

\section{Stärken des Systems}

Das System bietet feinräumige Auflösung (Scores für jeden Punkt statt ganzer
Städte), transparente Faktoraufschlüsselung, Personalisierbarkeit durch
dynamische Gewichtung, Echtzeitberechnung dank GiST-Indizes,
ausschließliche Nutzung offener Daten und plattformübergreifende Verfügbarkeit.

\section{Limitierungen}

Das System ist räumlich auf Bremen beschränkt. Die Vollständigkeit der
\acrshort{osm}-Daten variiert je nach Kategorie (vgl.\
Abschnitt~\ref{sec:osm_datenqualitaet}). Echtzeitdaten (Luftqualität,
Lärmpegel) fließen nicht ein. Die Basisgewichtungen basieren auf
Erfahrungswerten statt empirischen Studien. Binäre Faktoren differenzieren
nicht nach Größe/Intensität. Unfalldaten werden nur für ein Jahr importiert.

\section{Verbesserungspotenzial}
\label{sec:verbesserungen}

Auf Basis der identifizierten Limitierungen lassen sich mehrere
Weiterentwicklungen ableiten:

\begin{itemize}
      \item \textbf{Skalierung}: Generalisierung der Bounding-Box-Konfiguration
            und Erweiterung auf weitere deutsche Städte oder den gesamten DACH-Raum.
      \item \textbf{Echtzeitdaten}: Integration von Echtzeit-APIs für
            Luftqualität (z.\,B.\ Umweltbundesamt), Lärmkarten und
            ÖPNV-Verspätungen.
      \item \textbf{Machine Learning}: Erlernung optimaler Gewichtungen aus
            Nutzerfeedback (implizit durch Favoriten, explizit durch Bewertungen).
      \item \textbf{Differenzierte negative Faktoren}: Abstufung der Strafpunkte
            basierend auf der Distanz zum negativen Objekt (\textit{distance
                  decay}).
      \item \textbf{Aggregierte Unfalldaten}: Import und Mittelung über
            mehrere Jahrgänge zur Glättung von Ausreißern.
      \item \textbf{Offene API}: Bereitstellung einer öffentlichen API für
            Drittanwendungen und Forschungszwecke.
\end{itemize}
