% !TEX root = ../main.tex
% ============================================================
\chapter{Einleitung}
\label{ch:einleitung}
% ============================================================

\section{Motivation}

Die Wahl des Wohnortes ist eine der weitreichendsten Entscheidungen im Alltag.
Faktoren wie die Nähe zu Grünflächen, die Erreichbarkeit von Nahversorgung und
öffentlichem Nahverkehr, aber auch potenzielle Belastungen durch Lärm, Verkehr
oder Industrieanlagen beeinflussen die Lebensqualität eines Standortes erheblich.
Obwohl zahlreiche Geodaten zu diesen Aspekten frei verfügbar sind -- insbesondere
über \acrfull{osm} und den Unfallatlas des Statistischen Bundesamtes -- fehlt es
an Werkzeugen, die diese heterogenen Datenquellen standortbezogen aggregieren und
in einem leicht verständlichen Index zusammenfassen.

Internationale Lebensqualitätsrankings wie der \textit{Global Liveability Index}
der Economist Intelligence Unit \parencite{economist_gla} oder der \textit{Quality
      of Living Index} von Mercer \parencite{mercer2019} bewerten Städte auf nationaler
oder globaler Ebene, bieten jedoch keine Auflösung auf Stadtteil- oder gar
Adressebene. Hier setzt das vorliegende Projekt an.

\section{Zielsetzung}

Ziel des Projekts \textbf{Bremen Livability Index} (\textit{BLI}) ist die
Entwicklung einer vollständigen Geodaten\-verarbeitungs\--Pipeline, die:

\begin{enumerate}[label=\arabic*.]
      \item räumliche Daten aus \acrshort{osm} und dem Unfallatlas automatisiert in
            eine PostGIS-Datenbank importiert,
      \item für einen beliebigen Standort innerhalb Bremens einen
            \textbf{Livability Score} (0\,--\,100) in Echtzeit berechnet,
      \item den Score in positive und negative Einflussfaktoren aufschlüsselt,
      \item die umliegenden \acrshort{gis}-Features als GeoJSON-Objekte zur
            Visualisierung auf einer interaktiven Karte bereitstellt und
      \item dem Nutzer die Möglichkeit gibt, individuelle Gewichtungen
            (\textit{Präferenzen}) für die einzelnen Faktoren festzulegen.
\end{enumerate}

Das System wird als produktionstaugliche Web-, Mobile- und Desktop-Applikation
mit Flutter-Frontend und FastAPI-Backend realisiert.
Quellcode: \url{https://github.com/Milad9A/Bremen-Livability-Index} --
Webanwendung: \url{https://bremen-livability-frontend.onrender.com}

\section{Abgrenzung}

Die vorliegende Arbeit beschränkt sich räumlich auf das Gebiet der Freien
Hansestadt Bremen (Bounding-Box $53{,}0\text{\textdegree}\,\text{N}$ --
$53{,}2\text{\textdegree}\,\text{N}$,
$8{,}5\text{\textdegree}\,\text{E}$ --
$9{,}0\text{\textdegree}\,\text{E}$). Eine Übertragung auf andere Städte ist
konzeptionell möglich, wird jedoch nicht umgesetzt. Es werden ausschließlich frei
verfügbare, statische Datenquellen verwendet; Echtzeitdaten (z.\,B.\ aktuelle
Lärmwerte, Luftqualität) werden nicht berücksichtigt.

\section{Aufbau der Arbeit}

Nach einer Einführung in Grundlagen und Datenquellen
(Kapitel~\ref{ch:grundlagen}--\ref{ch:datenquellen}) werden Systemarchitektur,
Datenbankdesign und Bewertungsmethodik
(Kapitel~\ref{ch:architektur}--\ref{ch:bewertung}) erläutert. Die automatisierte
Datenerfassung (Kapitel~\ref{ch:datenerfassung}) sowie die Implementierung von
Backend-\acrshort{api} und Frontend
(Kapitel~\ref{ch:backend}--\ref{ch:frontend}) werden anschließend beschrieben.
Kapitel~\ref{ch:testing} behandelt Testing und Deployment; abschließend folgen
Ergebnisdiskussion und Fazit (Kapitel~\ref{ch:ergebnisse}--\ref{ch:fazit}).
